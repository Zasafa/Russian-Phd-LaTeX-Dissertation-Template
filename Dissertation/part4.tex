\chapter{Поглощение света многослойной наночастицей} \label{chapt4}

\section{Введение}

Экстинкция электромагнитной волны, падающей на наночастицу,
определяется суммой её рассеяния и поглощения. Некоторые аспекты
управления рассеянием с помощью многослойных покрытий были рассмотрены
в предыдущей главе.  В настоящей главе будет рассмотрен вопрос о
фундаментальных ограничениях на эффективность поглощения
электромагнитного излучения уединённой наночастицей.

Ранее уже предпринимались попытки рассмотрения этого вопроса. В своей
работе~\cite{Tribelsky-2011} М.И. Трибельский показал, что существует
верхний предел, ограничивающий возможности поглощения для одного
мультиполя (моды). Коэффициенты Ми для поглощения электрическими
$\tilde{a}_n= {\rmfamily Re}\{a_n\} - |a_n|^2 $ и магнитными
$\tilde{b}_n= {\rmfamily Re}\{b_n\} - |b_n|^2 $ модами не могут
превысить значения $1/4$. Здесь $n=1$ соответствует случаю диполя,
$n=2$ квадрупольной моде, и так далее для больших значений $n$.

Аналогичным образом, применяя только аналитические методы, Григорьев и
др.~\cite{Grigoriev-2015} рассмотрели эффект идеального поглощения на
примере двухслойной частицы, а именно случай диэлектрической сферы,
покрытой слоем золота или серебра. В дипольном приближении они
получили выражение для эффективного значения диэлектрической
проницаемости, обеспечивающего максимально достижимое поглощение в
сфере заданного размера.  В рамках теории эффективной среды
Максвелла-Гарнета ими было получено выражение для объёмной доли
диэлектрика, при которой возникает эффект идеального поглощения в
рассматриваемых двухслойных частицах.  Расчёт по этим выражениям
хорошо совпадает с расчётом по выражениям из работы
М.И. Трибельского~\cite{Tribelsky-2011} для случая $n=1$. Однако
указанный подход обладает рядом недостатков, среди которых можно
отметить ограниченную применимость (вследствие использования
дипольного приближения можно рассматривать только относительно
маленькие частицы) и то, что для большинства значений параметра
размера объёмная доля выражается комплексным числом.  Последний факт
делает эту теорию малопригодной для экспериментальной верификации.

Достоинством теории Ми является используемое разложение поля по
сферическим векторным гармоникам, что позволяет разделить вклад в
общее поле от электрической и магнитной дипольной моды, а также вклад
от квадруполей и мультиполей более высокого порядка. Таким образом,
становится возможен покомпонентный анализ спектрального отклика
многослойной сферы в зависимости от её дизайна. Например, в ряде
случаев удаётся совместить в спектре рассеяния положение нескольких
резонансов (таких как электрических дипольного и квадрупольного). Это
создаёт эффект суперрассеяния~\cite{Fan-2010,Fan-2011}, когда
многослойная частица специального дизайна рассеивает сильнее, чем
однородная частица того же размера из произвольного изотропного
материала.

В данной главе рассматривается аналогичный эффект суперпоглощения,
когда сечение поглощения сферической наночастицы превышает
фундаментальный предел поглощения резонансно возбуждённого мультиполя
максимального порядка.  При совмещении нескольких резонансов
становится возможным преодолеть фундаментальный предел, существующий
для одной моды.  В этом случае, аналогично эффекту суперрассеяния,
сечение поглощения оказывается больше, чем у однородной частицы того
же размера из произвольного изотропного материала.

Причина для этого проста: в случае однородной сферы нет возможности
совместить на одной частоте, например, несколько электрических
мод. Для заданного размера сферы из одного материала различие в
пространственной структуре низших мод приводит к тому, что они
оказываются существенно различны по частоте. Моды высокого порядка
(например моды шепчущей галлереи) могут быть оказаться разнесены по
частотам значительно меньше, чем соответствующая им ширина резонанса,
однако такие частицы оказываются гораздо больше, перестают быть
наноразмерными и не рассматриваются в настоящей работе. Аналогичным
образом оказывается невозможным совмещение на одной частоте
нескольких магнитных мод.

Последнее возможное сочетание мод для случая однородной сферы, это
совмещение электрического и магнитного резонанса. Однако для
материалов, чаще всего используемых при изготовлении наночастиц,
положение низших электрических и магнитных мод как правило не
совпадает по частоте.   




При исследовании поглощения света наночастицами рассматривались
трёхслойные частицы из заранее выбранных материалов, поэтому в
качестве параметров оптимизации использовались толщины составных
слоёв.  Целью исследования было получение структур с максимальной
эффективностью поглощения $Q_{\rmfamily sca} = C_{\rmfamily abs}/\pi R^2$ для
заданной длины волны, где $C_{\rmfamily abs}$ это сечение поглощения, а $R$
это внешний радиус наночастицы.  Результат оптимизации для различных
значений $R$ представлен на рисунках~\ref{img:q-abs}(а--в). На
рисунке~\ref{img:q-abs}(а) дополнительными пунктирными линиями
отмечены максимально достижимые эффективности поглощения для
дипольного ($n=1$) и квадрупольного ($n=2$) резонансов, выраженные в
виде~\cite{Tribelsky-2011}
\[Q^{(n)}_{\rmfamily  abs\ max}=\frac{2n+1}{2q^2}\]
через параметр размера $q=2\pi R/\lambda$.  В рассматриваемой системе
максимальный порядок резонансного возбуждения мультиполей ограничен
квадруполем ($n=2$). Так как дизайны с внешним радиусом $R>60$~нм
демонстрируют большую эффективность поглощения, то выполняются условия
для режима суперпоглощения.

\begin{figure}[t]
  \begin{minipage}[ht]{0.495\linewidth}
    \centering{\includegraphics[height=1.35\linewidth]{2015-04-01-Qabs-SiAgSi-overview}}
  \end{minipage}
  \hfill
  \begin{minipage}[ht]{0.495\linewidth}
    \centering{\includegraphics[height=1.35\linewidth]{2015-04-01-SiAgSi-ab-spectra4}}
  \end{minipage}
  \caption{ (а--в) Результат оптимизации эффективности поглощения
    $Si/Ag/Si$ наночастицей в зависимости от её внешнего радиуса, (а)
    эффективность поглощения, (б) коэффициенты поглощения в разложении
    Ми, где $\tilde{a}_1$ и $\tilde{a}_2$ относятся к электрическим, а
    $\tilde{b}_1$ и $\tilde{b}_2$ к магнитным диполю и квадруполю, (в)
    толщины составных слоёв наночастиц, (г--е) спектры коэффициентов
    поглощения в разложении Ми для дизайнов, соответствующих локальным
    максимумам на рисунке~\ref{img:q-abs}а.}
  \label{img:q-abs}  
\end{figure}


На рисунке~\ref{img:q-abs}(б) изображены значения коэффициентов Ми для
поглощения различными модами, горизонтальной пунктирной линией
отмечено значение теоретически достижимого предела, равное
1/4. В случае небольшого размера частицы основной вклад в поглощение
даёт электрический диполь $\tilde{a}_1$.  При оптимизации дизайнов для
$R > 56.6$~нм оказалось, что дизайны с поглощением одновременно на
электрическом и магнитном диполях позволяют достичь большего общего
сечения поглощения, чем в случае поглощения только электрическим
диполем. Такое качественное изменение соответствует разрывам линий на
рисунках~\ref{img:q-abs}(б--в) и реализует режим суперпоглощения.
Необходимо отметить, что для небольшого диапазона размеров частицы
$80.7<R<82.1$~нм оптимальное поглощение обеспечивает использование
электрического дипольного $\tilde{a}_1$ и магнитного квадрупольного
$\tilde{b}_2$ резонансов, что приводит ещё к двум разрывам линий на
рисунках для соответствующих значений внешнего радиуса~$R$.

На рисунке~\ref{img:q-abs}(в) представлены толщины составных слоёв
наночастицы, полученные в результате оптимизации эффективности
поглощения.  Неожиданно оказалось, что дизайны с преобладающим
дипольным механизмом поглощения (т.е. для размеров частицы менее
56.6~нм) могут быть двух видов.  Чтобы получить наилучшее поглощение
для $R<46$~нм, хватает использования всего двух слоёв, при оптимизации
толщина внутреннего слоя исходного трёхслойного дизайна обратилась в
ноль.  При $R=46$~нм поглощение диполем почти достигает своего
теоретического предела ($\tilde{a}_1>0.249$), чтобы удерживать его
вблизи этого значения для больших значений $R$, оптимизатор начинает
наращивать толщину внутреннего кремниевого слоя.  В свою очередь, это
приводит к появлению слабого поглощения  $\tilde{a}_2$,
что, впрочем, не позволяет достичь режима суперпоглощения для $n=2$.

Для расчёта спектров на рисунках~\ref{img:q-abs}(г--е) были
использованы экспериментальные дисперсионные зависимости для
показателей преломления из работы\cite{palik-1997}. Спектры
коэффициентов поглощения в разложении Ми построены для дизайнов,
соответствующих локальным максимумам на
рисунке~\ref{img:q-abs}(а). Спектр для дизайна с внешним радиусом
$R=36$~нм на рисунке~\ref{img:q-abs}(г) подтверждает дипольный
характер поглощения с резонансом на выбранной для оптимизации длины
волны $\lambda=500$~нм.  Спектры дизайнов с максимумами поглощения для
$R=63$~нм и $R=81$~нм (рисунки~\ref{img:q-abs}(д) и~\ref{img:q-abs}(е)
соответственно) обладают типичной для суперпоглощения структурой с
вырождением нескольких резонансов. На этих спектрах присутствуют
дополнительные резонансы, которые, впрочем, расположены в значительном
отдалении от выбранной для оптимизации длины волны, поэтому их вклад в
общее поглощение мал.

Особо надо отметить, что для получения максимальной эффективности
поглощения вовсе не требуется режим суперпоглощения.  Из
рисунка~\ref{img:q-abs}(а) следует, что максимальная эффективность
соответствует малым размерам частицы, где значение коэффициента Ми для
поглощения заметно меньше теоретического предела.  Среди всех
рассмотренных структур двухслойная частица $Ag/Si$ с внешним радиусом
36~нм обладает максимальной эффективностью поглощения.  Для неё
сечение поглощения более чем 5 раз превысило её геометрическое
сечение. Дополнительным преимуществом таких частиц может являться то,
что они должны быть проще и дешевле в производстве по сравнению с
трёхслойными.  В то же время для частиц большего размера ($R>60$~нм
для рассмотренных материалов) максимальная эффективность получается в
режиме суперпоглощения.  Это может оказаться существенно для случая,
когда изготовление многослойных частиц меньшего размера недоступно по
какой-либо технологической причине.

\clearpage