\chapter{Поглощение света многослойной наночастицей} \label{chapt4}
Теоретическое исследование новых оптических свойств может быть
полностью аналитическим. Например, в работе Григорьева и
др.~\cite{Grigoriev-2015} рассматривается эффект идеального поглощения
на примере двухслойной частицы, в частности, случай диэлектрической
сферы во вмещающей среде, покрытой слоем золота или серебра. Используя
дипольное приближение они получили выражение для эффективного значения
диэлектрической проницаемости, обеспечивающего максимально достижимое
значение поглощения для сферы заданного размера.  Применяя теорию
эффективной среды Максвелла-Гарнета они также получили выражение для
объёмной доли диэлектрика, при которой возникает эффект идеального
поглощения в рассматриваемых двухслойных частицах.  Указанный подход
обладает рядом недостатков, среди которых можно отметить ограниченную
применимость (вследствие использования дипольного приближения можно
рассматривать только относительно небольшие частицы), и то,
что для большинства значений параметра размера объёмная доля
выражается комплексным числом.  Последний факт делает эту теорию
малопригодной для экспериментальной верификации. 

Среди других примеров полностью аналитического рассмотрения стоит
отметить работы Zh.~Ruan и  Sh.~Fan~\cite{Fan-2011,Fan-2010}




\underline{\textbf{В четвёртой главе}} рассматривается явление
поглощения света многослойной наночастицей.

Достоинством теории Ми является используемое разложение поля по
сферическим векторным гармоникам, что позволяет разделить вклад в
общее поле от электрического и магнитного дипольного резонанса, а
также вклад резонансов квадруполей и мультиполей более высокого
порядка. Таким образом, становится возможен покомпонентный анализ
спектрального отклика многослойной сферы в зависимости от её
дизайна. Например, в ряде случаев удаётся совместить в спектре
рассеяния положение нескольких резонансов (например, электрических
дипольного и квадрупольного), что создаёт эффект
суперрассеяния~\cite{Fan-2010,Fan-2011}.

В данной главе был рассмотрен аналогичный эффект суперпоглощения
для наночастицы $Si/Ag/Si$, когда сечение поглощения сферической
наночастицы превышает фундаментальный предел поглощения резонансно
возбуждённого мультиполя максимального порядка. Ранее в своей
работе~\cite{Tribelsky-2011} М.И. Трибельский показал, что существует
верхний предел, ограничивающий возможности поглощения для одного
мультиполя, коэффициенты Ми для поглощения электрическими
$\tilde{a}_n= {\rm Re}\{a_n\} - |a_n|^2 $ и магнитными
$\tilde{b}_n= {\rm Re}\{b_n\} - |b_n|^2 $ модами не могут превысить
значения $1/4$.  При совмещении нескольких резонансов становится
возможным преодолеть этот фундаментальный предел. В этом случае
сечение поглощения оказывается больше, чем у однородной частицы того
же размера из произвольного изотропного материала.

При исследовании поглощения света наночастицами исследовались
трёхслойные частицы из заранее выбранных материалов, поэтому в
качестве параметров оптимизации использовались толщины составных
слоёв.  Целью оптимизации было получение структур с максимальной
эффективностью поглощения $Q_{\rm sca} = C_{\rm abs}/\pi R^2$ для
заданной длины волны, где $C_{\rm abs}$ это сечение поглощения, а $R$
это внешний радиус наночастицы.  Результат оптимизации для различных
значений $R$ представлен на рисунках~\ref{img:q-abs}(а-в). На
рисунке~\ref{img:q-abs}(а) дополнительными пунктирными линиями
отмечены максимально достижимые эффективности поглощения для
дипольного ($n=1$) и квадрупольного ($n=2$) резонансов выраженные в
виде~\cite{Tribelsky-2011}
$$Q^{(n)}_{\rm  abs\ max}=\frac{2n+1}{2q^2}$$
через параметр размера $q=2\pi R/\lambda$.  В рассматриваемой системе
максимальный порядок резонансного возбуждения мультиполей ограничен
квадруполем ($n=2$). Так как дизайны с внешним радиусом $R>60$~нм
демонстрируют большую эффективность поглощения, то выполняются условия
для режима суперпоглощения.

\begin{figure}[t]
  \begin{minipage}[ht]{0.495\linewidth}
    \center{\includegraphics[height=1.35\linewidth]{2015-04-01-Qabs-SiAgSi-overview}}
  \end{minipage}
  \hfill
  \begin{minipage}[ht]{0.495\linewidth}
    \center{\includegraphics[height=1.35\linewidth]{2015-04-01-SiAgSi-ab-spectra4}}
  \end{minipage}
  \caption{ (а-в) Результат оптимизации эффективности поглощения
    $Si/Ag/Si$ наночастицей в зависимости от её внешнего радиуса, (а)
    эффективность поглощения, (б) коэффициенты поглощения в разложении
    Ми, где $\tilde{a}_1$ и $\tilde{a}_2$ относятся к электрическим, а
    $\tilde{b}_1$ и $\tilde{b}_2$ к магнитным диполю и квадруполю, (в)
    толщины составных слоёв наночастиц, (г-е) спектры коэффициентов
    поглощения в разложении Ми для дизайнов, соответствующих локальным
    максимумам на рисунке~\ref{img:q-abs}а.}
  \label{img:q-abs}  
\end{figure}


На рисунке~\ref{img:q-abs}(б) изображены значения коэффициентов Ми для
поглощения различными модами, горизонтальной пунктирной линией
отмечено значение теоретически достижимого предела для них, равное
1/4. В случае небольшого размера частицы основной вклад в поглощение
даёт электрический диполь $\tilde{a}_1$.  При оптимизации дизайнов для
$R > 56.6$~нм оказалось, что дизайны с поглощением одновременно на
электрическом и магнитном диполях позволяет достичь большего общего
сечения поглощения, чем в случае поглощения только электрическим
диполем. Такое качественное изменение соответствует разрывам линий на
рисунках~\ref{img:q-abs}(б-в) и реализует режим суперпоглощения.
Необходимо отметить, что для небольшого диапазона размеров частицы
$80.7<R<82.1$~нм оптимальное поглощение обеспечивает использование
электрического дипольного $\tilde{a}_1$ и магнитного квадрупольного
$\tilde{b}_2$ резонансов, что приводит ещё к двум разрывам линий на
рисунках для соответствующих значений внешнего радиуса~$R$.

На рисунке~\ref{img:q-abs}(в) представлены толщины составных слоёв
наночастицы, полученные в результате оптимизации эффективности
поглощения.  Неожиданно оказалось, что дизайны с преобладающим
дипольным механизмом поглощения (т.е. для размеров частицы менее
56.6~нм) могут быть двух видов.  Чтобы получить наилучшее поглощение
для $R<46$~нм, хватает использования всего двух слоёв, при оптимизации
толщина внутреннего слой исходного трёхслойного дизайна обратилась в
ноль.  При $R=46$~нм поглощение диполем почти достигает своего
теоретического предела ($\tilde{a}_1>0.249$), чтобы удерживать его
вблизи этого значения для больших значений $R$, оптимизатор начинает
наращивать толщину внутреннего кремниевого слоя.  В свою очередь, это
приводит к появлению слабого поглощения  $\tilde{a}_2$,
что, впрочем, не позволяет достичь режима суперпоглощения для $n=2$.

Для расчёта спектров на рисунках~\ref{img:q-abs}(г-е) были
использованы экспериментальные дисперсионные зависимости для
показателей преломления из работы\cite{palik-1997}. Спектры
коэффициентов поглощения в разложении Ми построены для дизайнов,
соответствующих локальным максимумам на
рисунке~\ref{img:q-abs}(а). Спектр для дизайна с внешним радиусом
$R=36$~нм на рисунке~\ref{img:q-abs}(г) подтверждает дипольный
характер поглощения с резонансом на выбранной для оптимизации длины
волны $\lambda=500$~нм.  Спектры дизайнов с максимумами поглощения для
$R=63$~нм и $R=81$~нм (рисунки~\ref{img:q-abs}(д) и~\ref{img:q-abs}(е)
соответственно) обладают типичной для суперпоглощения структурой с
вырождением нескольких резонансов. На этих спектрах присутствуют
дополнительные резонансы, которые, впрочем, расположены в значительном
отдалении от выбранной для оптимизации длины волны, поэтому их вклад в
общее поглощение мал.

Особо надо отметить, что для получения максимальной эффективности
поглощения вовсе не требуется режим суперпоглощения.  Из
рисунка~\ref{img:q-abs}(а) следует, что максимальная эффективность
соответствует малым размерам частицы, где значение коэффициента Ми для
поглощения заметно меньше теоретического предела.  Среди всех
рассмотренных структур двухслойная частица $Ag/Si$ с внешним радиусом
36~нм обладает максимальной эффективностью поглощения.  Для неё
сечение поглощения более чем 5 раз превысило её геометрическое
сечение. Дополнительным преимуществом таких частиц может являться то,
что они должны быть проще и дешевле в производстве по сравнению с
трёхслойными.  В то же время для частиц большего размера ($R>60$~нм
для рассмотренных материалов) максимальная эффективность получается в
режиме суперпоглощения.  Это может оказаться существенно для случая,
когда изготовление многослойных частиц меньшего размера не доступно по
какой-либо технологической причине.

\clearpage