\chapter{Модификация теории Ми для случая многослойной сферы} \label{chapt1}

\section{Введение} \label{sect1_1}

\underline{\textbf{Первая глава}} посвящена модификации теории Ми для
случая многослойной сферы. 

Более 100 лет назад Густав Ми опубликовал свою оригинальную
работу~\cite{Mie-1908} о взаимодействии плоской электромагнитной волны
с однородной сферой.  Изложенная в ней теория впоследствии получила
его имя и в настоящее время входит в число основных инструментов
применяемых при анализе задач рассеяния и поглощения сферическими
объектами.  Несмотря на более чем вековую историю теории Ми, работы по
её дальнейшему развитию ведутся и в настоящее время% ~\cite{Suzuki-2008,
% MacKowski-2012, Lerme-2000, Xu-2005, Li-2006, Gogoi-2010,
% Santiago-2011}
.  Рядом авторов были предложены математические
модели~\cite{Yang-2003, Pena-scattnlay-2009}, позволяющие изучать
многослойные сферы с произвольным числом слоёв.  Основная сложность
при этом возникает при численной реализации этих моделей.  В теории Ми
решение для рассеянного поля выражается в виде разложения в ряд:
\begin{align*}
{\rm \mathbf{E}}_s &=\sum_{n=1}^{\infty} E_n \left( i a_n {\rm
    \mathbf{N}}_{e1n}^{(3)} - b_n{\rm\mathbf{M}_{o1n}^{(3)}} \right)\\
{\rm \mathbf{H}}_s &=\frac{k}{\omega\mu}
 \sum_{n=1}^{\infty} E_n \left( i b_n {\rm
    \mathbf{N}}_{o1n}^{(3)} + a_n{\rm\mathbf{M}_{e1n}^{(3)}} \right)  
\end{align*}
где $E_n=i^nE_0(2n+1)/n(n+1)$, $n$ порядок мультиполя, $E_0$ амплитуда
падающего поля, $a_n$ и $b_n$ коэффициенты разложения, соответствующие
электирческим и магнитным мультиполям, ${\rm \mathbf{N}}_{e1n}^{(3)}$,
${\rm \mathbf{N}}_{o1n}^{(3)}$, ${\rm\mathbf{M}_{o1n}^{(3)}}$ и
${\rm\mathbf{M}_{e1n}^{(3)}}$ это сферические векторные гармоники,
выражающиеся через тригонометрические функции, полимномы Лежандра и
сферические функции Бесселя и Ханкеля, $k$ и $\omega$ волновой вектор
и частота падающей волны, $\mu$ магнитная проницаемость в вакууме.
Поле внутри $l$-ого слоя стратифицированной сферы выражается
аналогичным образом~\cite{Yang-2003}:
\begin{align}
{\rm \mathbf{E}}_l &=\sum_{n=1}^{\infty} E_n \left(
                     c_n^{(l)}{\rm\mathbf{M}}_{o1n}^{(1)}
                     -i d_n^{(l)} {\rm \mathbf{N}}_{e1n}^{(1)}
                     +i a_n^{(l)} {\rm \mathbf{N}}_{e1n}^{(3)}
                     - b_n^{(l)}{\rm\mathbf{M}}_{o1n}^{(3)} 
                     \right)\label{eq:3p1}\\
{\rm \mathbf{H}}_l &=\frac{k_l}{\omega\mu} \sum_{n=1}^{\infty} E_n
                     \left(
                      d_n^{(l)}{\rm\mathbf{M}}_{e1n}^{(1)} 
                     +i c_n^{(l)} {\rm \mathbf{N}}_{o1n}^{(1)} 
                     -i b_n^{(l)} {\rm \mathbf{N}}_{o1n}^{(3)} 
                     - a_n^{(l)}{\rm\mathbf{M}}_{e1n}^{(3)} 
                     \right)\label{eq:3p2}  
\end{align}
где для каждого слоя определены коэффициенты разложения $d_n^{(l)}$ и
$c_n^{(l)}$ электрического и магнитного поля для входящего поля и,
аналогично, $a_n^{(l)}$ и $b_n^{(l)}$ для исходящего поля.  Связь
между всеми коэффициентами разложения можно выразить в виде системы
реккурентных уравнений, которые получаются из граничных условий между
слоями на напрерывность номальных компонент полей~\cite{Yang-2003}:
TODO move to diser 
\begin{equation} % \tag{S} % tag - вписывает свой текст
  \label{eq:A2d1}
    % \begin{multlined}
    \begin{alignedat}{2}
d^{(l+1)}_{n}m_{l} \psi^{\prime}_{n}&{\left (m_{l+1} x_{l} \right )}
- a^{(l+1)}_{n} m_{l} \zeta^{\prime}_{n}{\left (m_{l+1} x_{l} \right )}-\\
& - d^{(l)}_{n} m_{l+1} \psi^{\prime}_{n}{\left (m_{l} x_{l} \right )} 
+ a^{(l)}_{n} m_{l+1} \zeta^{\prime}_{n}{\left (m_{l} x_{l} \right )}
= 0
\end{alignedat}
\end{equation}
\begin{equation} % \tag{S} % tag - вписывает свой текст
  \label{eq:A2d2}
\begin{alignedat}{2}
c^{(l+1)}_{n} m_{l} \psi_{n}&{\left (m_{l+1} x_{l} \right )}
  - b^{(l+1)}_{n} m_{l} \zeta_{n}{\left (m_{l+1} x_{l} \right )}-\\
&- c^{(l)}_{n} m_{l+1} \psi_{n}{\left (m_{l} x_{l} \right )} 
+b^{(l)}_{n} m_{l+1} \zeta_{n}{\left (m_{l} x_{l} \right )}  =0
\end{alignedat}
\end{equation}
\begin{equation} % \tag{S} % tag - вписывает свой текст
  \label{eq:A2d3}
\begin{alignedat}{2}
c^{(l+1)}_{n} \psi^{\prime}_{n}&{\left (m_{l+1} x_{l} \right )}
- b^{(l+1)}_{n} \zeta^{\prime}_{n}{\left (m_{l+1} x_{l} \right )}-\\
&- c^{(l)}_{n} \psi^{\prime}_{n}{\left (m_{l} x_{l} \right )} 
+b^{(l)}_{n} \zeta^{\prime}_{n}{\left (m_{l} x_{l} \right )}   =0
\end{alignedat}
\end{equation}
\begin{equation} % \tag{S} % tag - вписывает свой текст
  \label{eq:A2d4}
\begin{alignedat}{2}
 d^{(l+1)}_{n} \psi_{n}&{\left (m_{l+1} x_{l} \right )}
- a^{(l+1)}_{n} \zeta_{n}{\left (m_{l+1} x_{l} \right )}-\\
& - d^{(l)}_{n} \psi_{n}{\left (m_{l} x_{l} \right )} 
+ a^{(l)}_{n} \zeta_{n}{\left (m_{l} x_{l} \right )}   =0
\end{alignedat}
% \end{multlined}
\end{equation}
где $m_l$ показатель преломления в слое, нормированный на показатель
преломления окружающего пространства, $x_l$ параметр размера внешнего
радиуса слоя, $\psi_{n}(z) = z j_n(z)$ и $\zeta_{n}(z) = z h_n^1(z)$
функции Риккати-Бесселя, выраженные через сферические функции Бесселя
и Ханкеля.  Из выражений для падающей и рассеянной волны получаются
дополнительные условия на коэффициенты разложения
$c_n^{(L+1)}=d_n^{(L+1)}=1$, $a_n=a_n^{(L+1)}$ и $b_n=b_n^{(L+1)}$,
где $L$ общее число слоёв. Так как внутри центального слоя $l=1$ нет
рассеяния и, соответственно, исходящего поля, то
$a_n^{(1)}=b_n^{(1)}=0$. Последнее условие является избыточным для
решения системы
уравнений~(\labelcref{eq:A2d1,eq:A2d2,eq:A2d3,eq:A2d4}) и поэтому оно
было использовано для дополнительной проверки самосогласованности
работы компьютерной программы.  После проведения необходимых
алгебраических преобразований значения коэффициентов разложения были
получены явно, в виде обратной реккурентной последовательности:
\begin{equation}
\label{eq:6p1}
a^{(l)}_n = \frac
{
    {D^{(1)}_{n}}{\left (m_{l} x_{l} \right )}
    T_1\left (m_{l+1} x_{l} \right )
    +
    T_3\left (m_{l+1} x_{l} \right )
    m_{l}/m_{l+1}
}
{
   \zeta_{n}\left (m_{l} x_{l} \right )
   U\left (m_{l} x_{l} \right )
}
\end{equation}
\begin{equation}
\label{eq:6p2}
b^{(l)}_n = \frac
{
    {D^{(1)}_{n}}{\left (m_{l} x_{l} \right )}
    T_2\left (m_{l+1} x_{l} \right )
    m_{l}/m_{l+1}
    +
    T_4\left (m_{l+1} x_{l} \right )
}
{
   \zeta_{n}\left (m_{l} x_{l} \right )
   U\left (m_{l} x_{l} \right )
}
\end{equation}
\begin{equation}
\label{eq:6p3}
c^{(l)}_n = \frac
{
    {D^{(3)}_{n}}{\left (m_{l} x_{l} \right )}
    T_2\left (m_{l+1} x_{l} \right )
    m_{l}/m_{l+1}
    +
    T_4\left (m_{l+1} x_{l} \right )
}
{
   \psi_{n}\left (m_{l} x_{l} \right )
   U\left (m_{l} x_{l} \right )
}
\end{equation}
\begin{equation}
\label{eq:6p4}
d^{(l)}_n = \frac
{
    {D^{(3)}_{n}}{\left (m_{l} x_{l} \right )}
    T_1\left (m_{l+1} x_{l} \right )
    +
    T_3\left (m_{l+1} x_{l} \right )
    m_{l}/m_{l+1}
}
{
   \psi_{n}\left (m_{l} x_{l} \right )
   U\left (m_{l} x_{l} \right )
}
\end{equation}
используя
\begin{equation*}
  U(z) =    {D^{(1)}_{n}}(z) - {D^{(3)}_{n}}(z)
\end{equation*}
\begin{equation*}
  T_1(z) =   a^{(l+1)}_{n}  \zeta_{n}(z) 
           - d^{(l+1)}_{n}  \psi_{n}(z)
\end{equation*}
\begin{equation*}
  T_2(z) =   b^{(l+1)}_{n}  \zeta_{n}(z) 
           - c^{(l+1)}_{n}  \psi_{n}(z)
\end{equation*}
\begin{equation*}
  T_3(z) =  d^{(l+1)}_{n}  D^{(1)}_{n}(z)  \psi_{n}(z) 
          - a^{(l+1)}_{n}  D^{(3)}_{n}(z)  \zeta_{n} (z)
\end{equation*}
\begin{equation*}
  T_4(z) =  b^{(l+1)}_{n}  D^{(1)}_{n}(z)  \psi_{n}(z) 
          - c^{(l+1)}_{n}  D^{(3)}_{n}(z)  \zeta_{n} (z)
\end{equation*}

где $D^{(1)}_{n} = \psi^{\prime}_{n}/\psi_{n}$ и
$D^{(3)}_{n} = \zeta^{\prime}_{n}/\zeta_{n}$ это логарифмические
производные функций Риккати-Бесселя. Подставляя
(\labelcref{eq:6p1,eq:6p2,eq:6p3,eq:6p4}) в уравнения (\ref{eq:3p1}) и
(\ref{eq:3p2}) можно вычислить величину электрического и магнитного
поля внутри и снаружи многослойной сферы. Дополнительно, выразив
сферческие векторные гармоники через логарифмические производные
функций Риккати-Бесселя, удалось заметно увеличить численную
устойчивость вычислений.

Используя полученные выражения была полностью переработанна
программа~\cite{Scattnlay-web}. В результате программа получила
возможность расчёта полей, доработка уже существоваших алгоритмических
решений позволило сократить время расчёта в 2.2 раза.




Мы можем сделать \textbf{жирный текст} и \textit{курсив}.

%\newpage
%============================================================================================================================

\section{Ссылки} \label{sect1_2}
Сошлёмся на библиографию. Одна ссылка: \cite[с.~54]{Sokolov}\cite[с.~36]{Gaidaenko}. Две ссылки: \cite{Sokolov,Gaidaenko}. Много ссылок:  \cite[с.~54]{Lermontov,Management,Borozda} \cite{Lermontov,Management,Borozda,Marketing,Constitution,FamilyCode,Gost.7.0.53,Razumovski,Lagkueva,Pokrovski,Sirotko,Lukina,Methodology,Encyclopedia,Nasirova,Berestova,Kriger}. И ещё немного ссылок: \cite{Article,Book,Booklet,Conference,Inbook,Incollection,Manual,Mastersthesis,Misc,Phdthesis,Proceedings,Techreport,Unpublished}. \cite{medvedev2006jelektronnye, CEAT:CEAT581, doi:10.1080/01932691.2010.513279,Gosele1999161,Li2007StressAnalysis, Shoji199895,test:eisner-sample,AB_patent_Pomerantz_1968,iofis_patent1960}

%Попытка реализовать несколько ссылок на конкретные страницы для стандартной реализации:[\citenum{Sokolov}, с.~54; \citenum{Gaidaenko}, с.~36].

%Несколько источников мультицитата \cites[vii--x, 5, 7]{Sokolov}[v--x, 25, 526]{Gaidaenko} поехали дальше

Ссылки на собственные работы:~\cite{vakbib1, confbib1}

Сошлёмся на приложения: Приложение \ref{AppendixA}, Приложение \ref{AppendixB2}.

Сошлёмся на формулу: формула \eqref{eq:equation1}.

Сошлёмся на изображение: рисунок \ref{img:knuth}.

%\newpage
%============================================================================================================================

\section{Формулы} \label{sect1_3}

Благодаря пакету \textit{icomma}, \LaTeX~одинаково хорошо воспринимает в качестве десятичного разделителя и запятую ($3,1415$), и точку ($3.1415$).

\subsection{Ненумерованные одиночные формулы} \label{subsect1_3_1}

Вот так может выглядеть формула, которую необходимо вставить в строку по тексту: $x \approx \sin x$ при $x \to 0$.

А вот так выглядит ненумерованая отдельностоящая формула c подстрочными и надстрочными индексами:
\[
(x_1+x_2)^2 = x_1^2 + 2 x_1 x_2 + x_2^2
\]

При использовании дробей формулы могут получаться очень высокие:
\[
  \frac{1}{\sqrt{2}+
  \displaystyle\frac{1}{\sqrt{2}+
  \displaystyle\frac{1}{\sqrt{2}+\cdots}}}
\]

В формулах можно использовать греческие буквы:
\[
\alpha\beta\gamma\delta\epsilon\varepsilon\zeta\eta\theta\vartheta\iota\kappa\lambda\\mu\nu\xi\pi\varpi\rho\varrho\sigma\varsigma\tau\upsilon\phi\varphi\chi\psi\omega\Gamma\Delta\Theta\Lambda\Xi\Pi\Sigma\Upsilon\Phi\Psi\Omega
\]

%\newpage
%============================================================================================================================

\subsection{Ненумерованные многострочные формулы} \label{subsect1_3_2}

Вот так можно написать две формулы, не нумеруя их, чтобы знаки равно были строго друг под другом:
\begin{align}
  f_W & =  \min \left( 1, \max \left( 0, \frac{W_{soil} / W_{max}}{W_{crit}} \right)  \right), \nonumber \\
  f_T & =  \min \left( 1, \max \left( 0, \frac{T_s / T_{melt}}{T_{crit}} \right)  \right), \nonumber
\end{align}

Выровнять систему ещё и по переменной $ x $ можно, используя окружение \verb|alignedat| из пакета \verb|amsmath|. Вот так: 
\[
    |x| = \left\{
    \begin{alignedat}{2}
        &&x, \quad &\text{eсли } x\geqslant 0 \\
        &-&x, \quad & \text{eсли } x<0
    \end{alignedat}
    \right.
\]
Здесь первый амперсанд  означает выравнивание по~левому краю, второй "--- по~$ x $, а~третий "--- по~слову <<если>>. Команда \verb|\quad| делает большой горизонтальный пробел. 

Ещё вариант:
\[
    |x|=
    \begin{cases}
    \phantom{-}x, \text{если } x \geqslant 0 \\
    -x, \text{если } x<0
    \end{cases}
\]

Можно использовать разные математические алфавиты:
\begin{align}
\mathcal{ABCDEFGHIJKLMNOPQRSTUVWXYZ} \nonumber \\
\mathfrak{ABCDEFGHIJKLMNOPQRSTUVWXYZ} \nonumber \\
\mathbb{ABCDEFGHIJKLMNOPQRSTUVWXYZ} \nonumber
\end{align}

Посмотрим на систему уравнений на примере аттрактора Лоренца:

\[ 
\left\{
  \begin{array}{rl}
    \dot x = & \sigma (y-x) \\
    \dot y = & x (r - z) - y \\
    \dot z = & xy - bz
  \end{array}
\right.
\]

А для вёрстки матриц удобно использовать многоточия:
\[ 
\left(
  \begin{array}{ccc}
  	a_{11} & \ldots & a_{1n} \\
  	\vdots & \ddots & \vdots \\
  	a_{n1} & \ldots & a_{nn} \\
  \end{array}
\right)
\]


%\newpage
%============================================================================================================================
\subsection{Нумерованные формулы} \label{subsect1_3_3}

А вот так пишется нумерованая формула:
\begin{equation}
  \label{eq:equation1}
  e = \lim_{n \to \infty} \left( 1+\frac{1}{n} \right) ^n
\end{equation}

Нумерованых формул может быть несколько:
\begin{equation}
  \label{eq:equation2}
  \lim_{n \to \infty} \sum_{k=1}^n \frac{1}{k^2} = \frac{\pi^2}{6}
\end{equation}

Впоследствии на формулы (\ref{eq:equation1}) и (\ref{eq:equation2}) можно ссылаться.

Сделать так, чтобы номер формулы стоял напротив средней строки, можно, используя окружение \verb|multlined| (пакет \verb|mathtools|) вместо \verb|multline| внутри окружения \verb|equation|. Вот так:
\begin{equation} % \tag{S} % tag - вписывает свой текст 
  \label{eq:equation3}
    \begin{multlined}
        1+ 2+3+4+5+6+7+\dots + \\ 
        + 50+51+52+53+54+55+56+57 + \dots + \\ 
        + 96+97+98+99+100=5050 
    \end{multlined}
\end{equation}

Используя команду \verb|\labelcref| из пакета \verb|cleveref|, можно
красиво ссылаться сразу на несколько формул
(\labelcref{eq:equation1,eq:equation3,eq:equation2}), даже перепутав
порядок ссылок \verb|(\labelcref{eq:equation1,eq:equation3,eq:equation2})|.
