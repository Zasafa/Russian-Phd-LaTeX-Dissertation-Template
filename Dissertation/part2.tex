\chapter{Метод стахостической оптимизации при решении обратной задачи
  теории Ми} \label{chapt2}

\section{Сравнение методов стахостической оптимизации}
\label{sec:construct-review}

Развитие нанофотоники, как и многих других разделов физики, происходит
при одновременном развитии теории и эксперимента, которые оказываются
тесно связаны между собой.  Эксперимент определяет меру правильности
теоретических построений и, в то же время, является источником новых
физических эффектов. В последнем случае, определяющей становится роль
теоретического анализа, необходимого для выделения сути явления из
вороха возможных побочных факторов.  Аналогично, при исследовании
оптических свойств наночастиц можно брать результаты эксперимента и
проверять, насколько они соотносятся с существующими моделями, либо
вначале теоретически находить дизайны с новыми физическими свойствами,
а уже потом проверять их экспериментально.  Учитывая высокую стоимость
и трудоёмкость эксперимента в области оптики наночастиц вариант с
предварительным теоретическим анализом проблемы оказывается более
привлекательным.

Теория Ми позволяет изучать целый ряд физических величин, в число
которых входят, например, сечение рассеяния, сечение
поглощения, распределение электромагнитного поля как внутри, так и
вблизи наночастицы.  Дополнительно в теории Ми есть возможность
разделять вклады связанные с дипольными резонансами, квадрупольными и
мультиполями большего порядка. Такое разнообразие сильно затрудняет
решение обратной задачи, где требуется определить дизайн наночастицы
для достижения заданных характеристик взаимодействия с падающей
волной. Кроме того, заданным параметрам взаимодействия падающей волны
и наночастицы может соответствовать несколько дизайнов или же ни
одного.

Попытка решать эту задачу полностью аналитическми сразу же приводит к
необходимости выбора тех характеристик взаимодействия поля с частицей,
чьи значения будут определять наиболее подходящий дизайн. Другими
словами, необходимо будет решать обратную задачу отдельно для сечения
рассеяния, отдельно для коэффициента усиления поля и так далее, а это
многократно увеличивает объём небходимых работ. Более того,
использование только аналитического подхода заведомо оказывается
неприменимым при учёте экспериментально измеренных дисперсионных
зависимостей материальных параметров. Например, когда требуется
определить дизайн частицы из заданных металлов для получения нужных
рабочих характеристик в какой-то полосе частот и дополнительно
определить оптимальную длину волны внутри выбранного диапазона.  В
результате, наиболее универсальным представляется численное решение,
когда компьютерная программа выбирает параметры дизайна наночастицы
максимально приближенные к оптимальным.

Для численного решения обратную задачу теории Ми можно
переформулировать в общем виде. Будем рассматривать расчёт по теории
Ми в виде некой сложной функции, которая в контексте численного
решения называется целевой функций $f(\mathbf{x})$, в англоязычной литературе
используются теримины objective funtion или fitness function.  Целевая
функция получает вектор значений для входных параметров, который в
нашем случае представляет из себя список материальных параметров и
толщин для каждого слоя. Результат вычисления целевой функции
получается в виде скаляра, который при решении задачи Ми может быть
как значением некой физической величины, получаемой из расчёта, так и
некой искусственной величиной, характеризующей отклик системы в целом,
например, отношение сечений рассеяния и поглощения. Тогда обратную
задачу можно сформулировать, как поиск такого вектора входных
параметров для целевой функции, который позволял бы получить на выходе
результате значение, равное заранее заданной величине или ниболее
приближенное к ней. Таким образом, на этапе формулирования общего вида
задачи снимается вопрос о существовании её решения: если нет такого
входного вектора параметров, из которого можно было точно получить
заданное значение на выходе, то будет получено приблизительного
решение. Этого может оказаться достаточно для обеспечения потребностей
широкого круга практических задач.

При такой постановке вопроса становится возможным использовать
различные методы численной оптимизации, которые позволяют находить
положение эстремума у произвольной функции. Поиск вектора входных
параметров $\mathbf{x_t}$, который позволяет получить целевое
значение $y_t=f(\mathbf{x_t})$, сводится к поиску минимума для новой
целевой функции $\left|f(\mathbf{x})-y_t\right|$. 

При выборе конкретного метода оптимизации возникает сложность,
обусловленная огромным количеством методов и большим числом
модификаций каждого метода. При выборе метода применительно к задаче
Ми были использованны следующие предпосылки:
\begin{itemize}
\item Несмотря на то, что решение Ми является аналитическим и
  выражается в виде разложения в ряд по сферическим векторным
  гармоникам, одновременное нахождение производных для зависимости от
  радиуса и материального параметра оказывается громоздким даже в
  случае однородной сферы. Это тем более верно для случая
  произвольного числа сферических слоёв, где решение получается в виде
  рекуррентного соотношения.  Таким образом, метод оптимизации не
  должен требовать для своей работы нахождения значений производных
  оптимизируемой функции, что особено актуально в случае, когда
  одновременно оптимизируются и толщина, и показатель преломления
  каждого слоя.
\item Решение образовано осциллирующими функциями и, как следствие,
  будет содержать большое количество локальных экстремумов. Поэтому
  алгоритмы оптимизации, требующие особого отношения к подобным
  случаям, оказываются заведомо менее производительными.
\item Параметры оптимизации и оптимизируемая величина являются
  вещественными числами.
\end{itemize}

Всё вместе это приводит к необходимости исключить из рассмотрения
такие популярные методы, как метод наискорейшего спуска (требующий
вычисления градиента), симплекс-метод Нелдера--Мида (есть сложность с
локальными экстремумами) и аналогичные им. В результате, приходится
ограничить выбор стохастическими методами, среди которых наиболее
распространёнными являются генетические
алгоритмы~\cite{Goldberg-GA-1989}, методы роя
частиц~\cite{Kennedy-PSO-1995} и методы дифференциальной
эволюции~\cite{Storn-DE-first-1997}.  Все эти алгоритмы используют
метод <<проб и ошибок>>.  Несколько пробных решений, называемых
индивидами, генерируются случайным образом и многократно (итеративно)
улучшаются в надежде найти некое удовлетворительное решение. Качество
решения оценивается целевой функцией, которая должна быть
сформулирована в оптимизируемой задаче.  Полная группа индивидов
называется популяцией.  Состояние популяции на конкретном шаге
итерации называется поколением.  Переход между поколениями
осуществляется в соответствии с рядом относительно простых правил,
которые составляют сущность определённого алгоритма.

Генетические алгоритмы обычно рассматривают вещественные числа в виде
набора битов.  В отличие от них, методы роя частиц и методы
дифференциальной эволюции могут работать в непрерывном пространстве
вещественных входных параметров естественным образом (используя
возможность сложения и вычитания векторов пробных решений), что делает
их гораздо более удобными для решения физических и инженерных
задач.  Производительность этих алгоритмов зависит от правильного
выбора значений некоторых внутренних параметров
алгоритма.  Использование адаптивных версий алгоритмов упрощает задачу
оптимизации: их значения внутренних параметров настраиваются
автоматически при переходе между поколениями. Как правило, адаптивным
алгоритмам нужно гораздо меньше (более чем на порядок) итераций, чем
неадаптивным, чтобы добиться того же результата оптимизации.

Сравнение~\cite{Gong-compare-EA-2014,Kang-compare-EA-RABC-2011}
адаптивного алгоритма дифференциальной эволюции
JADE~\cite{Jingqiao-JADE-2009} с адаптивной оптимизацией методом роя
частиц~\cite{Zhan-APSO-2008} и многими другими методами показало
превосходство JADE или результат сопоставимый с лучшими из числа
протестированных методов оптимизации для большинства стандартных
тестов~\cite{Schwefel-1981,Rosenbrock-1960,Muhlenbein-1991,back-1996,Griewank-1981}.
Такое преимущество и отсносительная простота алогритма JADE послужили
основанием для того, чтобы выбрать его в качестве основного алгоритма
оптимизации в настоящей работе. Дополнительно для алгоритма JADE была
реализована улучшенная скорость скрещивания (по алгоритму
PMCRADE~\cite{Li-PMCRADE-2011}), получившийся метод оптимизации был
применён в последующих главах диссертации.

\section{Реализация алгоритма JADE в виде программы}
\label{sec:jade}

Часть технических моментов, касающихся выбора языка программирования и
вопросов производительности итоговой программы, была изложена в
разделе~\ref{sec:code} и остаётся верной для реализации алгоритма
стохастической оптимизации. Тем не менее, есть ряд отличий и
дополнительных факторов.

Прежде всего это связано с тем, что на момент начала работ не было
обнаружено готового к применению программного кода алгоритма JADE, поэтому его
реализация была полностью выполнена автором настоящей диссертации. Это
позволило, во-первых, выбрать наиболее подходящий для реализации язык
программирования и, во-вторых, избежать возможных сложностей с
лицензированием чужого исходного кода. В частности, тип лицензии для
разработанной программы оптимизации был выбран полностью
совместимым с лицензией использования программы для расчётов по
теории Ми.

Важным является вопрос об использовании генераторов псевдослучайных
чисел (ГПСЧ). Для работы стохастического оптимизатора требуется
хороший источник энтропии, однако в стадартной библиотеке языка
Си\texttt{++} долгое время этому вопросу не уделялось должное
внимание. В стандарте языка Си\texttt{++}11, принятом в 2011 году, была
добавлена новая библиотека \verb+random+, реализующая целый ряд
ГПСЧ. Прежде всего, это Ranlux~\cite{Luscher-RNG-Ranlux-1994}, который
входит в число основных ГПСЧ для моделирования методом Монте-Карло, и
MT19937~\cite{Matsumoto-RNG-MT-1998}, использующий вихрь Мерсенна. В
последние годы MT19937 завоевал значительную популярность благодаря
своей высокой производительности и большому периоду при достаточно
хороших статистических показателях. Поэтому MT19937 является ГПСЧ
используемым по умолчанию в самых разнообразных программах, включая
широко известную Matlab Mathworks~\cite{Matlab-web}. Таким образом, для разработки
стахостического оптимизатора подходят только относительно недавние
стандарты языка, начиная с версии Cи\texttt{++}11 и более поздние.

Использование относительно свежей спецификации стандарта для языка
программирования Си\texttt{++} обладает дополнительными
преимуществами. Из возможностей, появившихся в Си\texttt{++}11,
активно использовался новый синтаксис для работы с перечисляемыми
коллекциями, механизм автоматического определения типа переменной на
этапе компиляции по результату задающего выражения и возможность
задания значений, используемых по умолчанию, для данных класса в
заголовочном файле. Всё вместе это заметным образом сокращает время,
необходимое для разработки новой функциональности и её дальнейшей отладки.

Современные процессоры, используемые для проведения расчётов, как
правило содержат несколько вычислительных ядер, что позволяет
одновременно выполнять несколько потоков вычислений. Так как расчёты
значений целевой функции для каждого индивида в рамках одного
поколения стохастической оптимизации не зависят друг от друга, то они
хорошо подходят для параллельного выполнения. Другая возможность в
полной мере задействовать вычислительные возможности современных
процессоров связана с случайной природой выполняемой
оптимизации. Итоговый результат оптимизации при одних и тех же
исходнных параметрах может различаться, что лучше всего заметно в
случае, когда есть несколько похожих по форме и значению локальных
минимумов, а глобальный минимум отсутствует. В этом случае финальное
значение, полученное при оптимизации, с приблизительно равной
вероятностью может оказаться в каждом из таких минимумов. Чтобы
определить возникновение подобной ситуации необходимо несколько
независимых запусков оптимизатора, каждый из которых может выполняться
на своём вычислительном ядре.

В настоящей работе для реализации был выбран последний вариант. Такой
выбор обоснован относительной простотой реализации и соотношением
типичного количества необходимых запусков оптимизатора при решении
задачи Ми (около десяти) и количеством вычислительных ядер у
современных процессоров в стационарных компьютерах (менее
десяти). Впрочем, предусмотрена возможность для реализации первого
варианта паралельного выполнения (когда в рамках одной оптимизации
вычислительные ядера используются для одновременного расчёта
нескольких значений целевой функции), который станет предпочтительным
при запуске оптимизатора на суперкомпьютерном кластере, где число
вычислительных ядер заметно превышает количество необходимых запусков
оптимизации, но ещё меньше или сравнимо с числом индивидов в поколении.

Также существует возможность гибрдного похода. В случае, когда
количество вычислительных ядер оказывается больше числа индивидов в
поколении, то их использование для одного запуска оптимизации
становится неэффективным из-за структуры информационных зависимостей в
алгоритме дифференциальной эволюции. Дело в том, что расчёт значений
целевой функции для очередного поколения оптимизации можно начинать
только после того, как будет завершён расчёт этих значений для
текущего поколения, будут проведен отбор лучших индивидов в поколении
и выполнена генарация индивидов для нового поколения. Поэтому,
например, в случае, когда число вычислительных ядер в два раза
превышает количество индивидов в поколении, то наиболее рациональным
является запуск двух независимых оптимизаций.

Выбранный алгоритм допускает и более сложные схемы балансировки,
которые могут быть реализованы в случае необходимости. Например, если
число ядер не кратно числу индивидов в поколении, то одним из
возможных вариантов является использование общей очереди
вычислений. Каждый из независимых запусков оптимизации размещает в ней
агруметны своих индивидов текущего поколения для расчёта целевой
функции.  Вычислительные ядера разбирают эту очередь, выполняют расчёт
и возвращают результат соответствующему процессу оптимизации. Такой
метод балансировки должен обеспечить высокую эффективность в случае,
когда число индивидов в текущем поколении в сумме по всем запускам
оптимизации заметно превышает количество вычислительных ядер.

Указанная выше особенность структуры информационных зависимостей в
алгоритме дифференциальной эволюции приводит к существованию верхнего
предела его эффективной параллелизации.  А именно, эффективно может быть
использовано число вычислительных ядер равное общему количеству
индивидов в текущем поколении всех запусков оптимизации. При
дальнейшем увеличении числа ядер величина параллельной эффективности
будет падать.  

Однако стоит учитывать, что, как правило, применение
суперкомпьютернных кластеров становится оправданным в случае
необходимости решения вычислительно трудоёмких задач, когда разовое
вычисление целевой функции само по себе требует заметного
времени. Поэтому в качестве независимого вычислителя, выполняющего
расчёт значения целевой функции, алгоритм оптимизации может вместо
одного ядра процессора рассматривать процессор целиком, вычислительный
узел или группу таких узлов. Общая параллельная эффективность при этом
будет определяться не только эффективностью оптимизатора, но и тем,
насколько эффективно может быть выполенено распараллеливание расчёта
целевой функции.  В качестве примера таких целевых функций можено
привести полноволновые расчёты трёхмерных моделей, использущих метод
конечных элементов.

Для реализации алгоритма, обеспечивающего параллельное выполнение
оптимизации, были выбраны программные библиотеки, поддерживающие
стандарт Message Parsing Interface (MPI). Это позволяет запускать
оптимизацию в несколько потоков как на одиночных процессорах, так и на
суперкомпьютерных кластерах. Несмотря на то, что на момент написания
диссертации поддерживался только самый простой вариант параллелизации
(количество используемых вычислительных ядер определяет число
независимых запусков оптимизиатора, выполняемых одновременно) подобная
совместимость с суперкомпьютерными кластерами оказалась удобной: на
одном кластере можно запускать несколько разных задач оптимизации,
каждая со своим числом запусков. Для этого через менеджер ресурсов и
задач кластера запрашивается необходимое число вычислительных ядер.

С помощью штатных возможностей функций MPI был реализован сбор
итоговых значений оптимизации по всем независимым запускам, после чего
был выполнялся расчёт статистических параметров: среднего значения и
среднеквадратичного отклонения. Это позволило удобным образом
огранизовать тестирование оптимизатора на ряде стандартных функций,
где по общепринятой методике проводится 50 запусков для каждой
тестовой функции. Более подробно вопрос тестирования оптимизатора
будет рассмотрен в разделе~\ref{sec:test-jade}.

В теории разработки программного обеспечения широко известны несколько
техник, которые позволяют создавать, отлаживать, развивать и
поддерживать сложные программные продукты.  Их использование позволяет
значительно снизить стоимость работы программиста, или, другими
словами, существенно упростить процесс разработки.  Недостатком ряда
подобных техник является наличие дополнительных накладных расходов на
их выполнение.  В случае, если создаваемая программа оказывается
недостаточно большой и сложной, то использование этих техник может не
окупиться в краткосрочной перспективе.  Часть техник оказываются
специфичны для каких-то языков программирования или выбранной
парадигмы (функциональной, императивной, объектно-ориентированной
и~т.д.), однако существует несколько общих принципов, применение
которых не требует больших затрат времени. В их число входят техника
сокрытия сложности за счёт разделения абстракций и техника
защитного программирования, которые были использованы при
разработке  оптимизатора в настоящей главе.

Сокрытие сложности во многом основано на более общем принципе,
согласно которому выполнение сложной задачи (которая не может быть
выполнена сразу и целиком) разбивается на несколько подзадач. В
терминах языка программирования Си\texttt{++} подзадачи называются
методами класса и функциями.  В общем случае при программировании
регламентируется количество подзадач и возникает требование
минимальной связности между ними.  Ограничение на количество подзадач
определяется особеностями человеческого восприятия. Если подзадач
становится более 10-15 штук, то резко возрастает сложность анализа их
взаимного влияния.  Эмпирически это ограничение широко известно как
<<правило одного экрана>>.  Так как при разработке декомпозиция
выражается в виде описания функции или метода класса, то не
рекомеднуется делать такое разбиение, которое занимает более 25 строк
в тексте программы, которые помещаются на экране стандартного
терминала.  С учётом строк, которые тратятся на формальное определение
функции, форматирование, логические конструкции, комментарии и тому
подобное, то в результате получается не более 10-15 логически
самостоятельных подзадач. На это число автор диссертаци и
ориентировался при реализации алгоритма стохастической оптимизации.


Более важным при сокрытии сложности программы является требование
минимальной связности между позадачами, что формально может быть
измерено числом логически обособленных единиц информации, необходимых
для выполнения подзадачи. Если, например, основная задача программы
состоит в создании, модификации и дальнейшей пересылки какой-то одной
структры данных, то только эти три подзадачи и надо включать в
декомпозию на первом уровне разбиения из-за их слабой связности. Если
подзадача оказалась сама по себе достаточно сложной, то она
подвергается дальнейшей декомпозиции, что приводит к возникновению
иерархии подпрограмм.  Число в 10-15 подзадач, указанные ранее, это
верхний предел, достижение которого сигнализирует о том, что какая-то
группа позадач скорее всего может быть выделена в отдельную подзадачу,
так как является более связанной. Например, формально это можно
понять, если такая группа позадач использует какие-то данные, которые
не используются несколькими подзадачами до и после неё.

При реализации уже существующего алгоритма дифференциальной эволюции
разбиение на подзадачи происходит естественным образом. Сам по себе
алгоритм в тексте программы был оформлен в виде отдельной логической
единицы, объединяющей в себе в общем виде данные и методы работы с
ними, необходимые для проведения оптимизации (в Си\texttt{++} такое
объединение называется классом). Среди методов класса, реализающих
ключевые шаги алгоритма оптимизации, можно отметить селекцию, мутацию,
кроссовер, адаптацию. В них используется ряд вспомогательных методов,
которые отвечают за генерацию нужных случайных распределений, выборку
индивидов для генерации следующего поколения и так далее. Ряд
интерфейсных методов, которые доступны для использования внешними
программами, отвечает за настройку параметров оптимизации,
инициализацию внутренних структур данных и запуск оптимизации в целом.

Указанные рекомендации часто используются программистами на уровне,
близком к интуитивному, и могут трактоваться довольно творчески.  Это
связано с тем, что всегда существуют ограничения по времени, которое
может быть потрачено на разработку. Поэтому в небольших проектах,
связанных с исследовательской деятельностью, более рациональным может
оказаться быстрое получение результата при достаточно плохой структуре
програамы, так как существенное увеличение расходов на поддержание и
дальнейшее развитие кода отсутствует, из-за того что он более не
используется. В случае реализации алгоритма стохастической оптимизации
это не так, возможная область применения такой программы довольно
широка. Поэтому вопросу структурирования разрабатываемой библиотеки
оптимизации было уделено достаточно много времени.  Опыт регулярного
использования библиотеки автором диссертации на протяжении почти 3-х
лет подтвердил правильность выбранного подхода.  В частности, когда
потребовалось добавить новую функционалость, а именно, возможность
задавать начальное значение для части инидивидов в популяции, это
удалось сделать без существенных затрат сил и времени, сохранив
обратную совместимость с программами, использовавшими предыдущую
версию оптимизатора.

На этапе проектирования программы оптимизатора был допущен ряд ошибок,
большая часть которых была исправлена на этапе отладки и
тестирования. Оставшиеся не влияют на корректность получаемых
результатов, однако несколько затрудняют использование оптимизатора.
Наиболее существенной среди них является выбор в пользу <<кодов
ошибок>> для отработки программой нештатных ситуаций вместо
использования <<механизма исключений>>.  Такое решение было принято в целях
экономии времени, так как начальное изучение методик применения
механиза исключений в Cи\texttt{++} потребовало неожиданно много
усилий, а коды ошибок являются хорошо проверенной и часто используемой
техникой при создании программ на родственном языке Си. Поэтому, среди
собственных переменных класса была объявленна специальная переменная
\verb+error_status_+, каждый метод в случае ошибки записывает туда
значение кода ошибки, соответствующее какому-то типу случившейся
нештатной ситуации.  В подавляющем большинстве случаев дальнейшее
выполнение программы можно прекратить, сообщив пользователю о типе
возникшей ошибке и месте её возникновения, что и происходит после
того, как ход выполнения программы доберётся до вызова соответствующих
инструкций. Недостатком этого подхода является дублирование
соответсвуюшего кода в теле различных методов класса, что увеличивает
общий объём кода, и, как следствие, затрудняет его дальнешее
усовершенстование.

Кроме того, применение кодов ошибок не позволяет в полной мере
разрешить ситуацию, когда проблема возникает в подпрограмме,
вычисляющей значение оптимизируемой функции. Дело в том, что эта
функция задаётся полностью независимо, поэтому, у неё нет доступа к
собственным переменным класса оптимизатора и она может, например,
использовать любые свои коды ошибок. Так как библиотека оптимизатора
должна быть универсальной, то возможно два варианта решения такой
коллизии. Во-первых, в документации можно объявить коды ошибок,
которые может возвращать оптимизируемая функция. Это ограничение не
всегда может быть выполненно в случае, когда такая функция является
уже готовй к использованию достаточно сложной
программой. Дополнительно, возникнет новая связь, существование
которой обусловлено техническими моментами и, как следствие нарушает
требование минимальной логической связности между частями программы:
явная передача кода ошибки никак не обоснована в рамках задачи
оптимизации.  Во-вторых, можно отказаться от кодов ошибок и
использовать механизм исключений, который существует на уровне языка
Си\texttt{++}.  В случае, если оптимизируемая функция уже использует
исключения для своей работы, такой подход представляется наиболее
естественным.  Возникновение неявная связи (создаваемое исключение
передаётся на более высокий уровень иерархии подпрограмм) частично
компенсируется однонаправленным характером такой связи и возможностью
исключить явную обработку исключений на некоторых уровнях иерархии.
Однако подобная возможность одновременно является и основным
недостаткой механизма исключений, так как генерация исключения и его
обработка могут оказаться разнесены по самым неожиданным частям
программы.  Это может привести к существенным затруднениям для
программиста при необходимости разобраться, что происходит в нештатной
ситуации, поэтому при применении механизма исключений в Си\texttt{++}
необходимо вдумчиво выбирать места для их обработки в тексте
программы.

При практическом применении оптимизатора для создания дизайнов
наночастицы стала понятна необходимость использования механизма
исключений. Дело в том, что в некоторых редких случаях входные
параметры, получаемые с помощью оптимизатора, приводили к
неустойчивому расчёту по теории Ми. Собщение о факте возникновения
неустойчивости передавалось по механизму исключений. Так как класс
оптимизатора был уже готов и использовал коды ошибок, то было принято
компромиссное решение. В результате, в настоящее время используются
оба подхода для разрешения нештатных ситуаций, однако в случае
необходимости дальнейшего улучшения оптимизатора запланированы работы
по миграции на использование только механизма исключений. Это позволит
увеличить степень унификации кода и упростит его разработку и
поддержание. 

\section{Тестирование реализации алгоритма JADE}
\label{sec:test-jade}
Максимум не факт что глобальный

\begingroup % Ограничиваем область видимости arraystretch
\needspace{2\baselineskip}
\renewcommand{\arraystretch}{1.6}%% Увеличение расстояния между рядами, для улучшения восприятия.
\begin{longtabu} to \textwidth {@{}>{\setlength{\baselineskip}{0.7\baselineskip}}X[1.1mc]>{\setlength{\baselineskip}{0.7\baselineskip}}X[mc]X[4]@{}}
        \caption{Тестовые функции для оптимизации, $D$ -
          размерность. Для всех функций значение в точке глобального
          минимума равно нулю.\label{tbl:test-functions}}\\% label всегда желательно идти после caption 
        
        \toprule     %%% верхняя линейка
        Имя           &Стартовый диапазон параметров &Функция  \\ 
        \midrule %%% тонкий разделитель. Отделяет названия столбцов. Обязателен по ГОСТ 2.105 пункт 4.4.5 
        \endfirsthead

        \multicolumn{3}{c}{\small\slshape (продолжение)}        \\ 
        \toprule     %%% верхняя линейка
        Имя           &Стартовый диапазон параметров &Функция  \\ 
        \midrule %%% тонкий разделитель. Отделяет названия столбцов. Обязателен по ГОСТ 2.105 пункт 4.4.5 
        \endhead
        
        \multicolumn{3}{c}{\small\slshape (окончание)}        \\ 
        \toprule     %%% верхняя линейка
        Имя           &Стартовый диапазон параметров &Функция  \\ 
        \midrule %%% тонкий разделитель. Отделяет названия столбцов. Обязателен по ГОСТ 2.105 пункт 4.4.5 
        \endlasthead

        \bottomrule %%% нижняя линейка
        \multicolumn{3}{r}{\small\slshape продолжение следует}  \\ 
        \endfoot   
        \endlastfoot

        сфера         &$\left[-100,\,100\right]^D$   &
        $\begin{aligned}\textstyle f_1(\mathbf{x})=\sum_{i=1}^Dx_i^2\end{aligned}$                                                        \\
        Schwefel 2.22 &$\left[-10,\,10\right]^D$     &
        $\begin{aligned}\textstyle f_2(\mathbf{x})=\sum_{i=1}^D|x_i|+\prod_{i=1}^D|x_i|\end{aligned}$                                     \\
        Schwefel 1.2  &$\left[-100,\,100\right]^D$   &$\begin{aligned}\textstyle f_3(\mathbf{x})=\sum_{i=1}^D\left(\sum_{j=1}^ix_j\right)^2\end{aligned}$                               \\
        Schwefel 2.21 &$\left[-100,\,100\right]^D$   &$\begin{aligned}\textstyle f_4(\mathbf{x})={\rmfamily max}_i\!\left\{\left|x_i\right|\right\}\end{aligned}$                             \\
        Rosenbrock    &$\left[-30,\,30\right]^D$     &$\begin{aligned}\textstyle f_5(\mathbf{x})=\sum_{i=1}^{D-1}\left[100\!\left(x_{i+1}-x_i^2\right)^2+(x_i-1)^2\right]\end{aligned}$ \\
        ступенчатая   &$\left[-100,\,100\right]^D$   &$\begin{aligned}\textstyle f_6(\mathbf{x})=\sum_{i=1}^D\big\lfloor x_i+0.5\big\rfloor^2\end{aligned}$                             \\ 
зашумлённая квартическая  &$\left[-1.28,\,1.28\right]^D$ &$\begin{aligned}\textstyle f_7(\mathbf{x})=\sum_{i=1}^Dix_i^4+rand[0,1)\end{aligned}$\vspace*{2ex}\\
        Schwefel 2.26 &$\left[-500,\,500\right]^D$   &$\begin{aligned}f_8(\mathbf{x})= &\textstyle\sum_{i=1}^D-x_i\,\sin\sqrt{|x_i|}\,+ \\
                    &\vphantom{\sum}+ D\cdot
                    418.98288727243369 \end{aligned}$\\
        Rastrigin     &$\left[-5.12,\,5.12\right]^D$ &
        $\begin{aligned}\textstyle
          f_9(\mathbf{x})=\sum_{i=1}^D\left[x_i^2-10\,\cos(2\pi
            x_i)+10\right]\end{aligned}$\vspace*{2ex}\\
  Ackley        &$\left[-32,\,32\right]^D$     &$\begin{aligned}f_{10}(\mathbf{x})= &\textstyle -20\, {\rmfamily exp}\!\left(-0.2\sqrt{\frac{1}{D}\sum_{i=1}^Dx_i^2} \right)-\\
                    &\textstyle - {\rmfamily exp}\left(\frac{1}{D}\sum_{i=1}^D\cos(2\pi x_i)  \right)  + 20 + e \end{aligned}$ \\
        Griewank      &$\left[-600,\,600\right]^D$
        &$\begin{aligned}f_{11}(\mathbf{x})= &\textstyle \frac{1}{4000}
          \sum_{i=1}^{D}x_i^2 - \prod_{i=1}^D\cos\left(x_i/\sqrt{i}\right) +1     \end{aligned}$ \vspace*{3ex} \\
        штрафная 1    &$\left[-50,\,50\right]^D$     &
        $\begin{aligned}f_{12}(\mathbf{x})= &\textstyle \frac{\pi}{D}
          \Big\{ 10\,\sin^2(\pi y_1) +\\ &+
          \textstyle \sum_{i=1}^{D-1}(y_i-1)^2\left[1+10\,\sin^2(\pi
              y_{i+1})\right] +\\ &+(y_D-1)^2 \Big\} +\textstyle\sum_{i=1}^D u(x_i,\,10,\,100,\,4)            \end{aligned}$ \vspace*{2ex} \\
        штрафная 2    &$\left[-50,\,50\right]^D$     &
        $\begin{aligned}f_{13}(\mathbf{x})= &\textstyle 0.1
          \Big\{\sin^2(3\pi x_1) +\\ &+
          \textstyle \sum_{i=1}^{D-1}(x_i-1)^2\left[1+\sin^2(3 \pi
              x_{i+1})\right] + \\ &+(x_D-1)^2\left[1+\sin^2(2\pi
              x_D)\right] \Big\} +\\ &+\textstyle\sum_{i=1}^D u(x_i,\,5,\,100,\,4)            \end{aligned}$            \vspace*{1ex}\\
        \midrule%%% тонкий разделитель
        \multicolumn{3}{@{}p{\textwidth}}{%
            \vspace*{-4ex}% этим подтягиваем повыше
            \hspace*{2.5em}% абзацный отступ - требование ГОСТ 2.105
            Примечание "---  Для функций $f_{12}$ и $f_{13}$
            используется $y_i = 1 + \frac{1}{4}(x_i+1)$ и
            $u(x_i,\,a,\,k,\,m)=\begin{cases}
k(x_i-a)^m,\quad &x_i >a\\[-0.5em]
0,\quad &-a\leq x_i \leq a\\[-0.5em]
k(-x_i-a)^m,\quad &x_i <-a
\end{cases}$  }   \\        \bottomrule %%% нижняя линейка 
\end{longtabu} \endgroup




\begingroup % Ограничиваем область видимости arraystretch
\needspace{2\baselineskip}
\renewcommand{\arraystretch}{1.6}%% Увеличение расстояния между рядами, для улучшения восприятия.
\begin{longtabu} to \textwidth {@{}>{\setlength{\baselineskip}{0.7\baselineskip}}X[1.1mc]>{\setlength{\baselineskip}{0.7\baselineskip}}X[mc]X[4]@{}}
        \caption{Сравнение различных алгоритмов оптимизации.\label{tbl:opt-results-book}}\\% label всегда желательно идти после caption 
        
        \toprule     %%% верхняя линейка
        Имя           &Стартовый диапазон параметров &Функция  \\ 
        \midrule %%% тонкий разделитель. Отделяет названия столбцов. Обязателен по ГОСТ 2.105 пункт 4.4.5 
        \endfirsthead

        \multicolumn{3}{c}{\small\slshape (продолжение)}        \\ 
        \toprule     %%% верхняя линейка
        Имя           &Стартовый диапазон параметров &Функция  \\ 
        \midrule %%% тонкий разделитель. Отделяет названия столбцов. Обязателен по ГОСТ 2.105 пункт 4.4.5 
        \endhead
        
        \multicolumn{3}{c}{\small\slshape (окончание)}        \\ 
        \toprule     %%% верхняя линейка
        Имя           &Стартовый диапазон параметров &Функция  \\ 
        \midrule %%% тонкий разделитель. Отделяет названия столбцов. Обязателен по ГОСТ 2.105 пункт 4.4.5 
        \endlasthead

        \bottomrule %%% нижняя линейка
        \multicolumn{3}{r}{\small\slshape продолжение следует}  \\ 
        \endfoot   
        \endlastfoot

        сфера         &$\left[-100,\,100\right]^D$   &
        $\begin{aligned}\textstyle f_1(\mathbf{x})=\sum_{i=1}^Dx_i^2\end{aligned}$                                                        \\
        Schwefel 2.22 &$\left[-10,\,10\right]^D$     &
        $\begin{aligned}\textstyle f_2(\mathbf{x})=\sum_{i=1}^D|x_i|+\prod_{i=1}^D|x_i|\end{aligned}$                                     \\
        Schwefel 1.2  &$\left[-100,\,100\right]^D$   &$\begin{aligned}\textstyle f_3(\mathbf{x})=\sum_{i=1}^D\left(\sum_{j=1}^ix_j\right)^2\end{aligned}$                               \\
        Schwefel 2.21 &$\left[-100,\,100\right]^D$   &$\begin{aligned}\textstyle f_4(\mathbf{x})={\rmfamily max}_i\!\left\{\left|x_i\right|\right\}\end{aligned}$                             \\
        Rosenbrock    &$\left[-30,\,30\right]^D$     &$\begin{aligned}\textstyle f_5(\mathbf{x})=\sum_{i=1}^{D-1}\left[100\!\left(x_{i+1}-x_i^2\right)^2+(x_i-1)^2\right]\end{aligned}$ \\
        ступенчатая   &$\left[-100,\,100\right]^D$   &$\begin{aligned}\textstyle f_6(\mathbf{x})=\sum_{i=1}^D\big\lfloor x_i+0.5\big\rfloor^2\end{aligned}$                             \\ 
зашумлённая квартическая  &$\left[-1.28,\,1.28\right]^D$ &$\begin{aligned}\textstyle f_7(\mathbf{x})=\sum_{i=1}^Dix_i^4+rand[0,1)\end{aligned}$\vspace*{2ex}\\
        Schwefel 2.26 &$\left[-500,\,500\right]^D$   &$\begin{aligned}f_8(\mathbf{x})= &\textstyle\sum_{i=1}^D-x_i\,\sin\sqrt{|x_i|}\,+ \\
                    &\vphantom{\sum}+ D\cdot
                    418.98288727243369 \end{aligned}$\\
        Rastrigin     &$\left[-5.12,\,5.12\right]^D$ &
        $\begin{aligned}\textstyle
          f_9(\mathbf{x})=\sum_{i=1}^D\left[x_i^2-10\,\cos(2\pi
            x_i)+10\right]\end{aligned}$\vspace*{2ex}\\
  Ackley        &$\left[-32,\,32\right]^D$     &$\begin{aligned}f_{10}(\mathbf{x})= &\textstyle -20\, {\rmfamily exp}\!\left(-0.2\sqrt{\frac{1}{D}\sum_{i=1}^Dx_i^2} \right)-\\
                    &\textstyle - {\rmfamily exp}\left(\frac{1}{D}\sum_{i=1}^D\cos(2\pi x_i)  \right)  + 20 + e \end{aligned}$ \\
        Griewank      &$\left[-600,\,600\right]^D$
        &$\begin{aligned}f_{11}(\mathbf{x})= &\textstyle \frac{1}{4000}
          \sum_{i=1}^{D}x_i^2 - \prod_{i=1}^D\cos\left(x_i/\sqrt{i}\right) +1     \end{aligned}$ \vspace*{3ex} \\
        штрафная 1    &$\left[-50,\,50\right]^D$     &
        $\begin{aligned}f_{12}(\mathbf{x})= &\textstyle \frac{\pi}{D}
          \Big\{ 10\,\sin^2(\pi y_1) +\\ &+
          \textstyle \sum_{i=1}^{D-1}(y_i-1)^2\left[1+10\,\sin^2(\pi
              y_{i+1})\right] +\\ &+(y_D-1)^2 \Big\} +\textstyle\sum_{i=1}^D u(x_i,\,10,\,100,\,4)            \end{aligned}$ \vspace*{2ex} \\
        штрафная 2    &$\left[-50,\,50\right]^D$     &
        $\begin{aligned}f_{13}(\mathbf{x})= &\textstyle 0.1
          \Big\{\sin^2(3\pi x_1) +\\ &+
          \textstyle \sum_{i=1}^{D-1}(x_i-1)^2\left[1+\sin^2(3 \pi
              x_{i+1})\right] + \\ &+(x_D-1)^2\left[1+\sin^2(2\pi
              x_D)\right] \Big\} +\\ &+\textstyle\sum_{i=1}^D u(x_i,\,5,\,100,\,4)            \end{aligned}$            \vspace*{1ex}\\
        \midrule%%% тонкий разделитель
        \multicolumn{3}{@{}p{\textwidth}}{%
            \vspace*{-4ex}% этим подтягиваем повыше
            \hspace*{2.5em}% абзацный отступ - требование ГОСТ 2.105
            Примечание "---  Для функций $f_{12}$ и $f_{13}$
            используется $y_i = 1 + \frac{1}{4}(x_i+1)$ и
            $u(x_i,\,a,\,k,\,m)=\begin{cases}
k(x_i-a)^m,\quad &x_i >a\\[-0.5em]
0,\quad &-a\leq x_i \leq a\\[-0.5em]
k(-x_i-a)^m,\quad &x_i <-a
\end{cases}$  }   \\        \bottomrule %%% нижняя линейка 
\end{longtabu} \endgroup


\section{Выводы}

Выполненная в рамках настоящей работы реализация стохастической
оптмизации методом адаптивной дифференциальной эволюции позволяет
эффективно использовать современные процессоры с большим количеством
параллельных потоков вычисления и может выполняться на
суперкомпьютерных кластерах.  Разработанное~\cite{JADE-web} программное обеспечение
успешно проходит набор стандартных тестов для алгоритмов оптимизации.

 Было получено
свидетельство о государственной регистрации программы для
ЭВМ~№2014611568.
\clearpage