\chapter{Метод стахостической оптимизации при решении обратной задачи
  теории Ми} \label{chapt2}

\underline{\textbf{Вторая глава}} посвящена выбору универсального
алгоритма оптимизации и вопросам его практической
реализации. Сложность выбора обусловлена огромным количеством методов
оптимизации, а также большим числом разновидностей каждого
метода. При выборе метода применительно к задаче Ми были использованны
следующие предпосылки:
\begin{itemize}
\item Несмотря на то, что решение Ми является аналитическим и
  выражается в виде разложения в ряд по сферическим векторным
  гармоникам, одновременное нахождение производных для зависимости от
  радиуса и материального параметра оказывается громоздким даже в
  случае однородной сферы, что тем более верно для случая
  произвольного числа сферических слоёв.  В связи с чем метод
  оптимизации не должен требовать для своей работы нахождения значений
  производных оптимизируемой функции.  Это особено актуально в случае,
  когда одновременно оптимизируются и толщина, и показатель
  преломления каждого слоя, или, например, оптимизируемая величина
  берётся в нескольких точках спектра одновременно.  Более того, могут
  возникать задачи, когда требуется оптимизировать значение поля внутри
  или рядом с конструируемой частицей.
\item Решение образовано быстро-осциллирующими функциями и, как
  следствие, будет содержать большое количество локальных
  экстремумов. Таким образом, алгоритмы оптимизации, требующие особого
  отношения к подобным случаям, оказываются заведомо менее
  производительными.
\item Параметры оптимизации (толщина и показатель
  преломления каждого слоя), а так же оптимизируемая величина
  (например эффективность рассеяния) являются
  вещественными числами.
\end{itemize}

Всё вместе это приводит к необходимости исключить из рассмотрения
такие популярные методы, как метод наискорейшего спуска (требующий
вычисления градиента), симплекс-метод Нелдера--Мида (есть сложность с
локальными экстремумами) и аналогичные им. В результате приходится
ограничить выбор стохастическими методами, среди которых наиболее
распространёнными являются генетические алгоритмы, методы роя частиц и
методы дифференциальной эволюции.  Все эти алгоритмы используют метод
<<проб и ошибок>>.  Несколько пробных решений (называемых индивидами)
генерируются случайным образом и многократно улучшаются в надежде
найти некое удовлетворительное решение. Качество решения оценивается
целевой функцией, формулируемой в задаче, которую предстоит
оптимизировать.  Полная группа индивидов называется популяцией.
Состояние популяции на конкретном шаге итерации называется поколением.
Переход между поколениями осуществляется в соответствии с рядом
относительно простых правил, которые составляют сущность определённого
алгоритма.

Генетические алгоритмы обычно рассматривают вещественные числа в виде
набора битов.  В отличие от них, методы роя частиц и методы
дифференциальной эволюции могут работать в непрерывном пространстве
вещественных входных параметров естественным образом (используя
возможность сложения и вычитания векторов пробных решений), что делает
их гораздо более удобными для решения физических и инженерных
задач.  Производительность этих алгоритмов зависит от правильного
выбора значений некоторых внутренних параметров
алгоритма.  Использование адаптивных версий алгоритмов упрощает задачу
оптимизации: их значения внутренних параметров настраиваются
автоматически при переходе между поколениями. Как правило, адаптивным
алгоритмам нужно гораздо меньше (более чем на порядок) итераций, чем
неадаптивным, чтобы добиться того же результата оптимизации.

В настоящей работе был реализован алгоритм JADE+ с улучшенной
скоростью скрещивания (по алгоритму PMCRADE), который является
адаптивным вариантом алгоритма дифференциальной эволюции. Он имеет
явное преимущество перед адаптивной оптимизацией методом роя частиц в
ряде стандартных тестов.  Выполненная в рамках настоящей работы
реализация указанного метода позволяет эффективно использовать
современные процессоры с большим количеством параллельных потоков
вычисления и может выполняться на суперкомпьютерных кластерах, что
стало возможно благодаря использованию программных библиотек,
поддерживающих стандарт Message Parsing Interface (MPI).
Разработанное~\cite{JADE-web} программное обеспечение успешно проходит
набор стандартных тестов для алгоритмов оптимизации.  Было получено
свидетельство о государственной регистрации программы для
ЭВМ~№2014611568.
\clearpage