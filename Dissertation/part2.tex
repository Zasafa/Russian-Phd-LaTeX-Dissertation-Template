\chapter{Метод стахостической оптимизации при решении обратной задачи
  теории Ми} \label{chapt2}

\section{Современные методы теоретического исследования новых
  дизайнов наночастиц}
\label{sec:construct-review}





\underline{\textbf{Вторая глава}} посвящена выбору универсального
алгоритма оптимизации и вопросам его практической
реализации. Сложность выбора обусловлена огромным количеством методов
оптимизации, а также большим числом разновидностей каждого
метода. При выборе метода применительно к задаче Ми были использованны
следующие предпосылки:
\begin{itemize}
\item Несмотря на то, что решение Ми является аналитическим и
  выражается в виде разложения в ряд по сферическим векторным
  гармоникам, одновременное нахождение производных для зависимости от
  радиуса и материального параметра оказывается громоздким даже в
  случае однородной сферы, что тем более верно для случая
  произвольного числа сферических слоёв.  В связи с чем метод
  оптимизации не должен требовать для своей работы нахождения значений
  производных оптимизируемой функции.  Это особено актуально в случае,
  когда одновременно оптимизируются и толщина, и показатель
  преломления каждого слоя, или, например, оптимизируемая величина
  берётся в нескольких точках спектра одновременно.  Более того, могут
  возникать задачи, когда требуется оптимизировать значение поля внутри
  или рядом с конструируемой частицей.
\item Решение образовано быстро-осциллирующими функциями и, как
  следствие, будет содержать большое количество локальных
  экстремумов. Таким образом, алгоритмы оптимизации, требующие особого
  отношения к подобным случаям, оказываются заведомо менее
  производительными.
\item Параметры оптимизации (толщина и показатель
  преломления каждого слоя), а так же оптимизируемая величина
  (например эффективность рассеяния) являются
  вещественными числами.
\end{itemize}

Всё вместе это приводит к необходимости исключить из рассмотрения
такие популярные методы, как метод наискорейшего спуска (требующий
вычисления градиента), симплекс-метод Нелдера--Мида (есть сложность с
локальными экстремумами) и аналогичные им. В результате приходится
ограничить выбор стохастическими методами, среди которых наиболее
распространёнными являются генетические алгоритмы, методы роя частиц и
методы дифференциальной эволюции.  Все эти алгоритмы используют метод
<<проб и ошибок>>.  Несколько пробных решений (называемых индивидами)
генерируются случайным образом и многократно улучшаются в надежде
найти некое удовлетворительное решение. Качество решения оценивается
целевой функцией, формулируемой в задаче, которую предстоит
оптимизировать.  Полная группа индивидов называется популяцией.
Состояние популяции на конкретном шаге итерации называется поколением.
Переход между поколениями осуществляется в соответствии с рядом
относительно простых правил, которые составляют сущность определённого
алгоритма.

Генетические алгоритмы обычно рассматривают вещественные числа в виде
набора битов.  В отличие от них, методы роя частиц и методы
дифференциальной эволюции могут работать в непрерывном пространстве
вещественных входных параметров естественным образом (используя
возможность сложения и вычитания векторов пробных решений), что делает
их гораздо более удобными для решения физических и инженерных
задач.  Производительность этих алгоритмов зависит от правильного
выбора значений некоторых внутренних параметров
алгоритма.  Использование адаптивных версий алгоритмов упрощает задачу
оптимизации: их значения внутренних параметров настраиваются
автоматически при переходе между поколениями. Как правило, адаптивным
алгоритмам нужно гораздо меньше (более чем на порядок) итераций, чем
неадаптивным, чтобы добиться того же результата оптимизации.

В настоящей работе был реализован алгоритм JADE+ с улучшенной
скоростью скрещивания (по алгоритму PMCRADE), который является
адаптивным вариантом алгоритма дифференциальной эволюции. Он имеет
явное преимущество перед адаптивной оптимизацией методом роя частиц в
ряде стандартных тестов.  Выполненная в рамках настоящей работы
реализация указанного метода позволяет эффективно использовать
современные процессоры с большим количеством параллельных потоков
вычисления и может выполняться на суперкомпьютерных кластерах, что
стало возможно благодаря использованию программных библиотек,
поддерживающих стандарт Message Parsing Interface (MPI).
Разработанное~\cite{JADE-web} программное обеспечение успешно проходит
набор стандартных тестов~\cite{Schwefel-1981,Rosenbrock-1960,Muhlenbein-1991,back-1996,Griewank-1981} для алгоритмов оптимизации. 

\begingroup % Ограничиваем область видимости arraystretch
\renewcommand{\arraystretch}{1.6}%% Увеличение расстояния между рядами, для улучшения восприятия.
\begin{longtabu} to \textwidth {@{}>{\setlength{\baselineskip}{0.7\baselineskip}}X[1.1mc]>{\setlength{\baselineskip}{0.7\baselineskip}}X[mc]X[4]@{}}
        \caption{Тестовые функции для оптимизации, $D$ -
          размерность. Для всех функций значение в точке глобального
          минимума равно нулю.\label{tbl:test-functions}}\\% label всегда желательно идти после caption 
        
        \toprule     %%% верхняя линейка
        Имя           &Стартовый диапазон параметров &Функция  \\ 
        \midrule %%% тонкий разделитель. Отделяет названия столбцов. Обязателен по ГОСТ 2.105 пункт 4.4.5 
        \endfirsthead

        \multicolumn{3}{c}{\small\slshape (продолжение)}        \\ 
        \toprule     %%% верхняя линейка
        Имя           &Стартовый диапазон параметров &Функция  \\ 
        \midrule %%% тонкий разделитель. Отделяет названия столбцов. Обязателен по ГОСТ 2.105 пункт 4.4.5 
        \endhead
        
        \multicolumn{3}{c}{\small\slshape (окончание)}        \\ 
        \toprule     %%% верхняя линейка
        Имя           &Стартовый диапазон параметров &Функция  \\ 
        \midrule %%% тонкий разделитель. Отделяет названия столбцов. Обязателен по ГОСТ 2.105 пункт 4.4.5 
        \endlasthead

        \bottomrule %%% нижняя линейка
        \multicolumn{3}{r}{\small\slshape продолжение следует}  \\ 
        \endfoot   
        \endlastfoot

        сфера         &$\left[-100,\,100\right]^D$   &
        $\begin{aligned}\textstyle f_1(x)=\sum_{i=1}^Dx_i^2\end{aligned}$                                                        \\
        Schwefel 2.22 &$\left[-10,\,10\right]^D$     &
        $\begin{aligned}\textstyle f_2(x)=\sum_{i=1}^D|x_i|+\prod_{i=1}^D|x_i|\end{aligned}$                                     \\
        Schwefel 1.2  &$\left[-100,\,100\right]^D$   &$\begin{aligned}\textstyle f_3(x)=\sum_{i=1}^D\left(\sum_{j=1}^ix_j\right)^2\end{aligned}$                               \\
        Schwefel 2.21 &$\left[-100,\,100\right]^D$   &$\begin{aligned}\textstyle f_4(x)={\rm max}_i\!\left\{\left|x_i\right|\right\}\end{aligned}$                             \\
        Rosenbrock    &$\left[-30,\,30\right]^D$     &$\begin{aligned}\textstyle f_5(x)=\sum_{i=1}^{D-1}\left[100\!\left(x_{i+1}-x_i^2\right)^2+(x_i-1)^2\right]\end{aligned}$ \\
        ступенчатая   &$\left[-100,\,100\right]^D$   &$\begin{aligned}\textstyle f_6(x)=\sum_{i=1}^D\big\lfloor x_i+0.5\big\rfloor^2\end{aligned}$                             \\ 
зашумлённая квартическая  &$\left[-1.28,\,1.28\right]^D$ &$\begin{aligned}\textstyle f_7(x)=\sum_{i=1}^Dix_i^4+rand[0,1)\end{aligned}$\vspace*{2ex}\\
        Schwefel 2.26 &$\left[-500,\,500\right]^D$   &$\begin{aligned}f_8(x)= &\textstyle\sum_{i=1}^D-x_i\,\sin\sqrt{|x_i|}\,+ \\
                    &\vphantom{\sum}+ D\cdot
                    418.98288727243369 \end{aligned}$\\
        Rastrigin     &$\left[-5.12,\,5.12\right]^D$ &
        $\begin{aligned}\textstyle
          f_9(x)=\sum_{i=1}^D\left[x_i^2-10\,\cos(2\pi
            x_i)+10\right]\end{aligned}$\vspace*{2ex}\\
  Ackley        &$\left[-32,\,32\right]^D$     &$\begin{aligned}f_{10}(x)= &\textstyle -20\, {\rm exp}\!\left(-0.2\sqrt{\frac{1}{D}\sum_{i=1}^Dx_i^2} \right)-\\
                    &\textstyle - {\rm exp}\left(\frac{1}{D}\sum_{i=1}^D\cos(2\pi x_i)  \right)  + 20 + e \end{aligned}$ \\
        Griewank      &$\left[-600,\,600\right]^D$
        &$\begin{aligned}f_{11}(x)= &\textstyle \frac{1}{4000}
          \sum_{i=1}^{D}x_i^2 - \prod_{i=1}^D\cos\left(x_i/\sqrt{i}\right) +1     \end{aligned}$ \vspace*{3ex} \\
        штрафная 1    &$\left[-50,\,50\right]^D$     &
        $\begin{aligned}f_{12}(x)= &\textstyle \frac{\pi}{D}
          \Big\{ 10\,\sin^2(\pi y_1) +\\ &+
          \textstyle \sum_{i=1}^{D-1}(y_i-1)^2\left[1+10\,\sin^2(\pi
              y_{i+1})\right] +\\ &+(y_D-1)^2 \Big\} +\textstyle\sum_{i=1}^D u(x_i,\,10,\,100,\,4)            \end{aligned}$ \vspace*{2ex} \\
        штрафная 2    &$\left[-50,\,50\right]^D$     &
        $\begin{aligned}f_{13}(x)= &\textstyle 0.1
          \Big\{\sin^2(3\pi x_1) +\\ &+
          \textstyle \sum_{i=1}^{D-1}(x_i-1)^2\left[1+\sin^2(3 \pi
              x_{i+1})\right] + \\ &+(x_D-1)^2\left[1+\sin^2(2\pi
              x_D)\right] \Big\} +\\ &+\textstyle\sum_{i=1}^D u(x_i,\,5,\,100,\,4)            \end{aligned}$            \vspace*{1ex}\\
        \midrule%%% тонкий разделитель
        \multicolumn{3}{@{}p{\textwidth}}{%
            \vspace*{-4ex}% этим подтягиваем повыше
            \hspace*{2.5em}% абзацный отступ - требование ГОСТ 2.105
            Примечание "---  Для функций $f_{12}$ и $f_{13}$
            используется $y_i = 1 + \frac{1}{4}(x_i+1)$ и
            $u(x_i,\,a,\,k,\,m)=\begin{cases}
k(x_i-a)^m,\quad &x_i >a\\[-0.5em]
0,\quad &-a\leq x_i \leq a\\[-0.5em]
k(-x_i-a)^m,\quad &x_i <-a
\end{cases}$  }   \\        \bottomrule %%% нижняя линейка 
\end{longtabu} 
\endgroup





 Было получено
свидетельство о государственной регистрации программы для
ЭВМ~№2014611568.
\clearpage