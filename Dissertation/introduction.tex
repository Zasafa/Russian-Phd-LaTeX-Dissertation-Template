\chapter*{Введение}							% Заголовок
\addcontentsline{toc}{chapter}{Введение}	% Добавляем его в оглавление

\newcommand{\actuality}{}
\newcommand{\progress}{}
\newcommand{\aim}{{\textbf\aimTXT}}
\newcommand{\tasks}{\textbf{\tasksTXT}}
\newcommand{\novelty}{\textbf{\noveltyTXT}}
\newcommand{\influence}{\textbf{\influenceTXT}}
\newcommand{\methods}{\textbf{\methodsTXT}}
\newcommand{\defpositions}{\textbf{\defpositionsTXT}}
\newcommand{\reliability}{\textbf{\reliabilityTXT}}
\newcommand{\probation}{\textbf{\probationTXT}}
\newcommand{\contribution}{\textbf{\contributionTXT}}
\newcommand{\publications}{\textbf{\publicationsTXT}}

{\actuality} 
% Актуальность - необходимо уметь контролировать рассеяние и поглощение,
% есть невидимость. Добавить 5 ссылок. Актуально сделать маскирующие
% покрытие на основе диэлектриков. 

В последние годы появилось большое количество работ по
нанофотонике~\cite{Tame-quantum-plasmonics-2013,
  Javier-graphene-plasmonics-2014, Khurgin-loss-plasmonics-2015,
  He-tunable-terahertz-graphene-metamaterials-2015,
  Segal-meta-nonlinar-PhC-2015,
  Poddubny-hyperbolic-metamaterials-2013, Kildishev-metasurface-2013}.
Высокая актуальность полученных результатов связана с перспективами их
практического применения и обусловлена стремительным развитием
нанотехнологий, что даёт возможность экспериментальной проверки
предлагаемых идей и подходов. Среди прочих, стоит отметить вопрос о
взаимодействии света с многослойной сферической наночастицей. Он
рассматривается в ряде прикладных задач, таких как: лечение
рака~\cite{Zhang-2010, Hirsch-2003}, различные методы диагностики в
медицине~\cite{Allain-2002}, разработка маскирующих суб-волновых
покрытий для видимого и микроволнового диапазонов~\cite{Qui-2009,
  Semouchkina-2013}, устройства плазмоники~\cite{Martin-2013,
  Alu-2005}, изучение тепловых свойств изоляторов~\cite{Xie-2013},
повышение эффективности солнечных элементов~\cite{Kameya-2011,
  Mann-2011} и так далее. Всё вместе это обуславливает актуальность
настоящей работы, в которой сперва излагается общий принцип,
позволяющий управлять рассеянием и поглощением электромагнитных волн
многослойными сферическими наночастицами, а потом идёт апробация на
частных примерах: минимизация рассеяния от идеально проводящей сферы
(частичная невидимость) и управление поглощением плазмонной частицы
$Si/Ag/Si$.

\underline{\textbf{Основные методы исследования.}}
Теория Ми~\cite{Mie-1908} входит в число основных инструментов применяемых
при анализе задач рассеяния и поглощения плоской электромагнитной
волны сферическими объектами. Эта теория была обобщённа на случай
многослойной сферы с произвольным числом слоёв~\cite{Yang-2003,
  Pena-scattnlay-2009} и доработана в настоящей работе, что позволило
реализовать её в виде комплекса программ для проведения компьютерного
моделирования. Достоинством теории является используемое ей разложение
поля по сферическим векторным гармоникам, что позволяет разделить
вклад в общее поле от электрического и магнитного дипольного
резонанса, а так же вклад резонансов квадруполя
и мультиполей более высокого порядка. Таким образом, становится
возможен анализ спектрального отклика многослойной сферы в зависимости
от её параметров (размеров и показателей преломления слоёв). Например,
в ряде случаев удаётся совместить в спектре рассеяния положение
нескольких резонансов (например, электрических дипольного и
квадрупольного), что создаёт эффект
суперрассеяния~\cite{Fan-2010,Fan-2011}. Аналогичный эффект
суперпоглощения подробно рассмотрен в настоящей работе.


 Как правило, при их решении возникает
необходимость оптимизации дизайна многослойной сферы (радиусов и
материальных параметров составных слоёв), обеспечивающего наилучшие
рабочие характеристики для каждого конкретного случая с учётом
фактических ограничений в предметной области.


\aim\ данной работы является разработка общего подхода к оптимизации
дизайнов многослойных сфер в рамках теории Ми, его последующая
реализация в комплексе компьютерных программ, выявление
закономерностей между дизайном многослойной сферы и её оптическими
свойствами.

Для~достижения поставленной цели необходимо было решить следующие {\tasks}:
\begin{enumerate}
  \item Разработать алгоритм для вычисления рассеяния и поглощения в
    многослойных сферических объектах и реализовать его в комплексе программ.
  \item Выбрать и реализовать алгоритм оптимизации, подходящий для
    работы с произвольными параметрами модели, описываемой обобщённой
    теорией Ми.
  \item Выявить основные закономерности взаимодействия с
    электромагнитной волной сферических маскирующих покрытий на
    основе диэлектриков.
  \item Исследовать эффект суперпоглощения света в многослойных
    сферических наночастицах.
\end{enumerate}



\defpositions
\begin{enumerate}
  \item Получены и реализованны в комплексе программ явные
    реккурентные соотношения для коэффициентов Ми в объёме
    многослойной сферы, выраженные через логарифмические производные
    функций Риккати-Бесселя увеличивающие численную стабильность.  
  
  \item Использование тонкого (размер мишени к размеру покрытия) диэлектриких многослойных покрытий позволяет
    уменьшить рассеяние от идеальное мишени в два раза.
  \item Использовать диэлектрикого порытия для небольшого объекта
    позволяет уменьшить рассеяние в 6 раз.

  \item TODO Использование алгоритма стохастической оптимизации методом
    адаптивной дифференциальной эволюции для решения задачи Ми
    позволяет выявлять семейства дизайнов с заранее заданными
    электромагнитными свойствами.

  \item Защищать цифры (уменьшили в два раза и т.д.). Показано, что
    маскирующие сферические покрытия из диэлектриков могут быть
    сконструированы, используя волноводоподобный эффект.  В этом
    случае при распространении внутри покрытия поле отстает по фазе от
    невозмущённой падающей волны на величину, кратную $2\pi$.
  \item Обнаружено семейство маскирующих сферических порытий из
    диэлектрических изотропных метаматериалов, реализующих эффект
    волнового обтекания.  Для получения заметного эффекта достаточно
    трёх слоёв в покрытии.
  \item В трёхслойных частицах $Si/Ag/Si$ возможно вырождение
    резонансных мультипольных откликов, приводящее к эффекту
    суперпоглощения, когда сечение поглощение оказывается больше, чем
    у bulk частицы. 
  \end{enumerate}

Положения соответствуют пункту 1 паспорта специальности 01.04.05 --
<<Оптика>> (Волновая (физическая) оптика. Интерференция, дифракция,
поляризация, когерентность света) по физико-математическим
наукам (представлены результаты фундаментальных исследований).

%\vspace{5.5em}
\novelty Используем диэлектрики для маскировки, использовать
оптимизация. Есть суперпоглощения.
\begin{enumerate}
  \item Впервые были получены явные реккурентные соотношения для
    коэффициентов Ми в многослойной сфере, выраженные через
    логарифмические производные функций Риккати-Бесселя. 
  \item Впервые метод дифференциальной эволюции был применён
    для изучения маскирующих сферических покрытий, показана высокая
    производительность метода.
  \item Было выполнено оригинальное исследование поглощения света
    наночастицами в режиме вырождения резонансых мультипольных откликов.
\end{enumerate}

\influence. Разработанные аналитические и численные методы для решения
уравнений Максвелла в рамках теории Ми, а так же реализующий их
программный комплекс с использованием стахостической оптимизации
методом дифференциальной эволюции могут быть использованы при
проектировании, оптимизации и анализе (включая анализ предельно
достижимых рабочих характеристик) широкого спектра устройств,
работающих как в оптическом, так и микроволновом диапазоне. Результаты
полученные при изучении поглощени света наночастицами могут быть
использованы при разработке инновационных устройств наноплазмоники,
фотоактивных катализаторов, красителей, поглощающих эмульсий и
аэрозолей.

Результаты диссертационной работы использовались при выполнении
грантов Министерства образования и науки РФ
(проект 11.G34.31.0020, гос. задание 2014/190, задание 3.561.2014/K),
Правительства РФ (грант 074-U01), РФФИ (грант 15-57-45141 ИНД\verb+_+а).


\reliability\ полученных результатов обеспечивается методическим
подходом на каждом этапе работы. Работа оптимизатора была проверена на
наборе стандартных тестовых функций. Аналитические результаты работы
были проверены в системе компьютерной алгебры (IPython). Компьютерная
реализация решения была проверена на наборе тестовых
задач. Аналитические результаты находятся в соответствии с
результатами, полученными другими авторами по теории Ми для случаев
однородной сферы и сферы с одним слоем покрытия.  Случаи большего
числа слоёв в покрытии сравнивался с коммерческими пакетами
моделирования, использующих численные методы конечных разностей во
временной области (Lumerical FDTD), метод конечных элементов (Comsol)
и метод конечных интегралов (CST MWS). Результаты по исследованию
маскирующих покрытий и поглощения света наночастицами находятся в
соответствии с результатами, полученными другими авторами для похожих
систем.

\probation\
Основные результаты работы докладывались~на:
перечисление основных конференций, симпозиумов и~т.\:п.

\contribution\ Автор принимал активное участие \ldots

%\publications\ Основные результаты по теме диссертации изложены в ХХ печатных изданиях~\cite{Sokolov,Gaidaenko,Lermontov,Management},
%Х из которых изданы в журналах, рекомендованных ВАК~\cite{Sokolov,Gaidaenko}, 
%ХХ --- в тезисах докладов~\cite{Lermontov,Management}.
 
\ifthenelse{\equal{\thebibliosel}{0}}{% Встроенная реализация с загрузкой файла через движок bibtex8
    \publications\ Основные результаты по теме диссертации изложены в XX печатных изданиях, 
    X из которых изданы в журналах, рекомендованных ВАК, 
    X "--- в тезисах докладов.%
}{% Реализация пакетом biblatex через движок biber
%Сделана отдельная секция, чтобы не отображались в списке цитированных материалов
    \begin{refsection}%
        \printbibliography[heading=countauthornotvak, env=countauthornotvak, keyword=biblioauthornotvak, section=1]%
        \printbibliography[heading=countauthorvak, env=countauthorvak, keyword=biblioauthorvak, section=1]%
        \printbibliography[heading=countauthorconf, env=countauthorconf, keyword=biblioauthorconf, section=1]%
        \printbibliography[heading=countauthor, env=countauthor, keyword=biblioauthor, section=1]%
        \publications\ Основные результаты по теме диссертации изложены в \arabic{citeauthor} печатных изданиях\nocite{bib1,bib2}, 
        \arabic{citeauthorvak} из которых изданы в журналах, рекомендованных ВАК\nocite{Ladutenko-cloak-2014,Ladutenko-Qabs-2015}, 
        \arabic{citeauthorconf} "--- в тезисах докладов\nocite{DD-14, MW-14}.%
    \end{refsection}
}
% При использовании пакета \verb!biblatex! для автоматического подсчёта
% количества публикаций автора по теме диссертации, необходимо
% их здесь перечислить с использованием команды \verb!\nocite!.
    

 % Характеристика работы по структуре во введении и в автореферате не отличается (ГОСТ Р 7.0.11, пункты 5.3.1 и 9.2.1), потому её загружаем из одного и того же внешнего файла, предварительно задав форму выделения некоторым параметрам

\textbf{Объем и структура работы.} Диссертация состоит из~введения,
четырёх глав и заключения.
%% на случай ошибок оставляю исходный кусок на месте, закомментированным
%Полный объём диссертации составляет  \ref*{TotPages}~страницу с~\totalfigures{}~рисунками и~\totaltables{}~таблицами. Список литературы содержит \total{citenum}~наименований.
%
Полный объём диссертации составляет \formbytotal{TotPages}{страниц}{у}{ы}{} 
с~\formbytotal{totalcount@figure}{рисунк}{ом}{ами}{ами}
и~\formbytotal{totalcount@table}{таблиц}{ей}{ами}{ами}. Список литературы содержит  
\formbytotal{citenum}{наименован}{ие}{ия}{ий}.
