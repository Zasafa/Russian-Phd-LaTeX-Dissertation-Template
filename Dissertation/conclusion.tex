\chapter*{Заключение}						% Заголовок
\addcontentsline{toc}{chapter}{Заключение}	% Добавляем его в оглавление

В соответствии с целью настоящей диссертационной работы было выполнено
исследование рассеяния и поглощения электромагнитных волн
многослойными сферическими наночастицами.
%% Согласно ГОСТ Р 7.0.11-2011:
%% 5.3.3 В заключении диссертации излагают итоги выполненного исследования, рекомендации, перспективы дальнейшей разработки темы.
%% 9.2.3 В заключении автореферата диссертации излагают итоги данного исследования, рекомендации и перспективы дальнейшей разработки темы.
%% Поэтому имеет смысл сделать эту часть общей и загрузить из одного файла в автореферат и в диссертацию:
Основные результаты работы заключаются в следующем:
%% Согласно ГОСТ Р 7.0.11-2011:
%% 5.3.3 В заключении диссертации излагают итоги выполненного исследования, рекомендации, перспективы дальнейшей разработки темы.
%% 9.2.3 В заключении автореферата диссертации излагают итоги данного исследования, рекомендации и перспективы дальнейшей разработки темы.
\begin{enumerate}
  \item Предложен метод изучения экстремальных оптических свойств
    многослойных сферических наночастиц с помощью теории Ми и
    стохастической оптимизации. Высокая вычислительная
    производительность этого подхода позволила выявить несколько новых
    физических эффектов, связанных с рассеянием и поглощением
    электромагнитной волны на многослойных сферических наночастицах.
  \item В задаче рассеяния плоской волны на многослойной сфере
    получены явные рекуррентные соотношения для коэффициентов Ми в
    расчёте локальных полей, выраженные через логарифмические
    производные функций Риккати-Бесселя.  Эти соотношения были
    добавлены в компьютерную программу, выполняющую вычисления в
    рамках задачи Ми.
  \item Рассеяние от объекта из идеального проводника можно
    существенно уменьшить с помощью многослойного покрытия толщиной
    $0.15\lambda$, используя только изотропные диэлектрические
    материалы: в 2 и в 6 раз для объектов диаметром $1.5\lambda$ и
    $\lambda/1.5$ соответственно. Обнаружен пороговый характер
    уменьшения рассеяния в зависимости от толщины покрытия.
  \item % (TODO берем балошени без слов min-max-min)
    Среди разнообразных оптимизированных дизайнов маскирующих покрытий из
    изотропных материалов с $\varepsilon$ меньше единицы, состоящих из
    множества слоёв равной толщины, выявлена закономерность,
    позволяющая разрабатывать эффективные трёхслойные сферические
    покрытия с разными толщинами слоёв. Дополнительной особенностью
    таких покрытий является значительное увеличение области спектра, в
    которой наблюдается эффект маскировки, при сравнении с покрытиями
    из диэлектриков. 
    %%%%%% Спектр --- засада
  \item В трёхслойных частицах $Si/Ag/Si$ возможно вырождение
    мультипольных резонансов, приводящее к эффекту суперпоглощения,
    когда сечение поглощения оказывается больше, чем у однородной
    частицы того же размера из произвольного изотропного
    материала. Максимальная эффективность поглощения в
    рассматриваемой системе была получена для небольших двухслойных
    частиц с преобладающей ролью электрического дипольного резонанса.
\end{enumerate}

На каждом этапе выполняемых работ методично проверялась достоверность
получаемых результатов; предлагаемый подход к изучению оптических
свойств многослойных сферических наночастиц был верифицирован на ряде
дополнительных примеров.

Указанный подход в перспективе может быть применён при исследовании и
разработке высокоэффективных рассеивателей и поглотителей, обладающих
в случае необходимости селективными спектральными свойствами,
концентраторов поля с большим коэффициентом усиления, устройств
наноплазмоники, диэлектрических наноантенн с заранее заданными
свойствами диаграммы направленности и любых других систем, описываемых
с помощью теории Ми. Используя дополнительные аналитические модели,
можно расширить область применимости метода, например, для задач
оптоакустики, для расчёта оптических сил, для случая нескольких
частиц. Основное ограничение на возможности расширения носит
технический характер и связано с необходимостью многократного
выполнения расчёта целевой функции в процессе стохастической
оптимизации.

В заключение автор выражает благодарность и большую признательность
научному руководителю Белову~П.А. за поддержку, помощь, обсуждение
результатов и научное руководство. Также автор благодарит Ovidio
Pe\~{n}a-Rodr\'{i}guez из Политехнического университета Мадрида за
помощь при подготовке программы, выполняющей расчёты по теории Ми,
Шадривова~И.В. и Мирошниченко~А.Е. из Австралийского национального
университета за плодотворные обсуждения результатов четвёртой главы
диссертации, особая благодарность Богдановой~О.Ю. и авторам шаблона 
\textit{Russian-Phd-LaTeX-Dissertation-Template} за помощь в
оформлении диссертации. Автор также благодарит коллектив кафедры
нанофотоники и метаматериалов университета ИТМО, сотрудников ФТИ
им.~А.Ф.~Иоффе и всех, кто сделал настоящую работу автора возможной.
