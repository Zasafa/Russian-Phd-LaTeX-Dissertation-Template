\chapter*{Список сокращений и условных обозначений}             % Заголовок
\addcontentsline{toc}{chapter}{Список сокращений и условных обозначений}  % Добавляем его в оглавление
\noindent
\addtocounter{table}{-1}% Нужно откатить на единицу счетчик номеров таблиц, так как следующая таблица сделана для удобства представления информации по ГОСТ
%\begin{longtabu} to \dimexpr \textwidth-5\tabcolsep {r X}
\begingroup % Ограничиваем область видимости arraystretch
\renewcommand{\arraystretch}{1.2}%% Увеличение расстояния между рядами, для улучшения восприятия.
\begin{longtabu} to \textwidth {r X}

% Жирное начертание для математических символов может иметь
% дополнительный смысл, поэтому они приводятся как в тексте
% диссертации
$a_n,b_n$  & 
коэффициенты разложения Ми в дальнем поле соответствующие
электрическим и магнитным мультиполям
\\
$a_n^{(l)}, b_n^{(l)}$  & 
коэффициенты разложения Ми электрического и магнитного поля для исходящей волны (направленной из центра
многослойной частицы) внутри слоя $l$
\\
$d_n^{(l)},c_n^{(l)}$  & 
аналогично для входящей волны (направленной к центру частицы) 
\\
$\rmfamily \boldsymbol{A}$ & поле вектора Пойтинга\\
$D$ & размерность целевой функции\\ 
$\boldsymbol{\hat{\rmfamily e}}$ & единичный вектор \\
$\boldsymbol{\rmfamily E}, \boldsymbol{\rmfamily H}$ & напряжённости электрического и
магнитного поля\\
$E_0$ & амплитуда падающего поля\\
$f(\boldsymbol{x})$ & целевая функция\\
$i$ & мнимая единица или индекс суммирования\\
$j$ & индекс суммирования \\
$k$ & волновой вектор падающей волны\\
$l$ & номер слоя внутри стратифицированной сферы\\
$L$ & общее число слоёв\\
$\lambda$ & длина волны электромагнитного излучения
в вакууме\\
$m_l$ & показатель преломления в слое, нормированный на показатель
преломления окружающего пространства\\
$\mu$  & магнитная проницаемость в вакууме\\
$n$ & порядок мультиполя\\
$N_{\rmfamily max}, N_{\rmfamily stop}$ & число членов рекурсии, фактически используемое и
эмпирическое приближение\\
$j$ & тип функции Бесселя\\
$\begin{rcases}
{\mathbf{N}}_{e1n}^{(j)}&{\mathbf{N}}_{o1n}^{(j)}\\
{\mathbf{M}_{o1n}^{(j)}}&{\mathbf{M}_{e1n}^{(j)}}
\end{rcases}$  & сферические векторные гармоники\\
$\omega$ & частота падающей волны\\
$\pi_n, \tau_n$ & угловые функции\\
$P_n$ & полином Лежандра\\
$P_n^m$ & функция Лежандра\\
$z_n^{(j)}, j_n, h_n^1$ & сферические функции Бесселя и Ханкеля первого рода\\
$\psi_{n}, \zeta_{n}$ & функции Риккати-Бесселя\\
$D^{(j)}_{n}$ & логарифмические производные функций Риккати-Бесселя\\
$r,\theta,\phi$ & полярные координаты\\
$\rho$ & обезразмеренное расстояние до центра сферы\\
$r_l$ &  внешний радиус слоя\\
$\boldsymbol{\rmfamily S},  \boldsymbol{\bar{\rmfamily S}}$ & вектор плотности потока энерги
электромагнитного поля (Пойтинга) и вектор его среднего значения за период\\
$t$ & координата вдоль линии потока энергии\\
$\boldsymbol{x}, x_i, x_j$ & вектор параметров целевой функции и его компоненты\\
$x_l$ & параметр размера внешнего радиуса слоя\\
$y$ & результат вычисления целевой функции \\
$y_t$ & целевое значение оптимизации\\
$z$ & комплексный аргумент\\
\textbf{BEM} & boundary element method, метод граничных элементов\\
\textbf{CST MWS} & Computer Simulation Technology Microwave Studio
программа для компьютерного моделирования уравнений Максвелла\\
\textbf{DDA} & discrete dipole approximation, приближение дискретиных диполей\\
\textbf{FDFD} & finite difference frequency domain, метод конечных
разностей в частотной области\\
\textbf{FDTD} & finite difference time domain, метод конечных
разностей во временной области\\
\textbf{FEM} & finite element method,  метод конечных элементов\\
\textbf{FIT} & finite integration technique, метод конечных интегралов\\
\textbf{FMM} & fast multipole method, быстрый метод многополюсника\\
\textbf{FVTD} & finite volume time-domain, метод конечных объёмов во
временной области\\
\textbf{MLFMA} & multilevel fast multipole algorithm, многоуровневый
быстрый алгоритм многополюсника\\
\textbf{MoM} & method of moments, метод моментов\\
\textbf{MSTM} & multiple sphere T-Matrix, метод Т-матриц для множества
сфер\\
\textbf{PEC} & perfect electric conductor, идеальный электрический проводник\\
\textbf{PSTD} & pseudospectral time domain method, псевдоспектральный
метод во временной области \\
\textbf{PML} & perfectly matched layer, идеально поглощающий слой\\
\textbf{TLM} & transmission line matrix method, метод матриц линий
передач\\
\textbf{ГПСЧ} & генератор псевдослучайных чисел\\
\textbf{ПО} & программное обеспечение\\
\end{longtabu} \endgroup
