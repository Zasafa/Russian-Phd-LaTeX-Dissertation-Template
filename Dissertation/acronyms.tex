\chapter*{Список сокращений и условных обозначений}             % Заголовок
\addcontentsline{toc}{chapter}{Список сокращений и условных обозначений}  % Добавляем его в оглавление
%\begin{tabularx}{\dimexpr \textwidth-5\tabcolsep}{r X}
\noindent
\begin{tabularx}{\textwidth}{r X}
  \textbf{BEM} & boundary element method, метод граничных элементов\\
  \textbf{CST MWS} & Computer Simulation Technology Microwave Studio
  программа для компьютерного моделирования уравнений Максвелла\\
  \textbf{DDA} & discrete dipole approximation, приближение дискретиных диполей\\
  \textbf{FDFD} & finite difference frequency domain, метод конечных
  разностей в частотной области\\
\end{tabularx}
\textbf{FDTD} - finite difference time domain, метод конечных
разностей во временной области

\textbf{FEM} - finite element method,  метод конечных элементов

\textbf{FIT} - finite integration technique, метод конечных интегралов

\textbf{FMM} - fast multipole method, быстрый метод многополюсника

\textbf{FVTD} - finite volume time-domain, метод конечных объёмов во
временной области

\textbf{MLFMA} - multilevel fast multipole algorithm, многоуровневый
быстрый алгоритм многополюсника

\textbf{MoM} - method of moments, метод моментов

\textbf{MSTM} - multiple sphere T-Matrix, метод Т-матриц для множества сфер

\textbf{PSTD} - pseudospectral time domain method, псевдоспектральный
метод во временной области 

\textbf{TLM} - transmission line matrix method, метод матриц линий передач
 

\boldmath$\lambda$ - длина волны падающего электромагнитного излучения
в вакууме