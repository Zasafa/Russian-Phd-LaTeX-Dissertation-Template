{\actuality} Существование фундаментальных ограничений на скорость
обработки информации в традиционной электронике привело к бурному
развитию нанофотоники. В последние годы появилось большое количество
работ~\cite{Tame-quantum-plasmonics-2013,
  Javier-graphene-plasmonics-2014, Khurgin-loss-plasmonics-2015,
  He-tunable-terahertz-graphene-metamaterials-2015,
  Segal-meta-nonlinar-PhC-2015,
  Poddubny-hyperbolic-metamaterials-2013, Kildishev-metasurface-2013}
в этой области.  Высокая актуальность полученных результатов связана с
перспективами их практического применения и обусловлена стремительным
развитием нанотехнологий, что даёт возможность экспериментальной
проверки предлагаемых идей и подходов. Одной из важных компонент
нанофотоники являются сферические наночастицы.  Актуальность настоящей
работы, посвященной изучению многослойных сферических наночастиц,
обусловлена их применением для лечения рака~\cite{Zhang-2010,
  Hirsch-2003}, различных методов диагностики в
медицине~\cite{Allain-2002}, разработки маскирующих суб-волновых
покрытий для видимого и микроволнового диапазонов~\cite{Qui-2009,
  Semouchkina-2013}, устройств плазмоники~\cite{Martin-2013,
  Alu-2005}, изучения тепловых свойств изоляторов~\cite{Xie-2013},
повышения эффективности солнечных элементов~\cite{Kameya-2011,
  Mann-2011}.

% \underline{\textbf{Степень разработанности:}} TODO хороших инструметнов для
% изучения многослоек - мало, способов получать годные новые дизайны -
% того меньше.

\aim\ данной работы является % разработка общего подхода к оптимизации
% дизайнов многослойных сфер в рамках теории Ми, его последующая
% реализация в комплексе компьютерных программ, выявление
% закономерностей между дизайном многослойной сферы и её
исследование рассеяния и поглощения электромагнитных волн многослойными
сферическими наночастицами.

Для~достижения поставленной цели необходимо было решить следующие {\tasks}:
\begin{enumerate}
  \item Разработать алгоритм для вычисления рассеяния и поглощения в
    многослойных сферических объектах и реализовать его в комплексе программ.
  \item Выбрать и реализовать алгоритм оптимизации, подходящий для
    работы с произвольными параметрами модели% , описываемой обобщённой
    % теорией Ми
    .
  \item Выявить основные закономерности взаимодействия с
    электромагнитной волной сферических маскирующих покрытий на
    основе диэлектриков.
  \item Исследовать эффект суперпоглощения света в многослойных
    сферических наночастицах.
\end{enumerate}

\underline{\textbf{Методология и методы исследования:}} Одним из
основных инструментов, применяемых при анализе задач рассеяния и
поглощения плоской электромагнитной волны объектами со сферической
симметрией, является теория Ми~\cite{Mie-1908}, суть которой
сводится к разложению полей в ряд по сферическим векторным
гармоникам. Эта теория была обобщена на случай многослойной сферы с
произвольным числом слоёв~\cite{Yang-2003, Pena-scattnlay-2009} и
доработана в настоящей работе, что позволило реализовать её в виде
комплекса программ для проведения компьютерного моделирования.

TODO все в настоящем времени
Для изучения оптических свойств многослойной наночастицы требуется
решать обратную задачу, т.е. определять дизайн многослойной сферы
(радиусы и комплексные показатели преломления составных слоёв) для
заданных характеристик рассеяния или отражения. Для случая однородной
сферы подбная задача была частична решена, например, в дипольном
приближении были полученых выражения для показателя преломления
обеспечивающего максимально достижимое значения поглощения для сферы
заданного размера\cite{Grigoriev-2015}. Однако, для
общего случая многослойной сферы с учётом вклада мультиполей высших
порядков большое число параметров модели делает явное решение
труднореализуемым. Вместо этого в настоящей работе используется
метод стохастической оптимизации, а именно метод адаптивной
дифференциальной эволюции. Он позволяет численным образом проводить
оптимизацию дизайна многослойной сферы, обеспечивающего наилучшие
рабочие характеристики для каждого конкретного случая с учётом
фактических ограничений в выбранной предметной области.

%\vspace{5.5em}
\novelty   
%  Используем диэлектрики для маскировки, использовать
% оптимизация. Есть суперпоглощения.
\begin{enumerate}
  \item Получены явные реккурентные соотношения для
    коэффициентов Ми в многослойной сфере, выраженные через
    логарифмические производные функций Риккати-Бесселя. 
  \item Метод дифференциальной эволюции был применён для
    поиска дизйнов многослойной сферы в рамках теории Ми,
    показана высокая вычислительная производительность метода.
  \item Показно, что использование многослойного диэлектрического
    покрытия для маскировки сферы из идеального проводника диаметром
    $1.5\lambda$ может уменьшать сечение рассеяния более чем в два
    раза.
  \item Обнаружен эффект суперпоглощения света в многослойных
    наночастицах $Si/Ag/Si$.
    % TODO проверить отклик, что он не вырождается
\end{enumerate}




\defpositions
%Защищать цифры (уменьшили в два раза и т.д.).
\begin{enumerate}
  \item Получены явные реккурентные соотношения для коэффициентов Ми,
    выраженные через логарифмические производные функций Риккати-Бесселя
    и реализованны в комплексе
    программ. %, увеличивающие численную устойчивость вычислений.
  \item Рассеяние от объекта из идеального проводника может быть
    существенно уменьшено с помощью многослойного покрытия толщиной
    $0.15\lambda$ используя только изотропные диэлектрические
    материалы: в 2 и в 6 раз для объектов диаметром $1.5\lambda$ и
    $\lambda/1.5$ соответственно.
  \item Был обнаружен новый класс многослойных сферических покрытий,
    основанный на использовании материалов с $\varepsilon < 1$.
    %%%%%% Спектр- засада
  \item В трёхслойных частицах $Si/Ag/Si$ возможно вырождение
    мультипольных резонаносов, приводящее к эффекту
    суперпоглощения, когда сечение поглощения оказывается больше, чем
    у однородной частицы того же размера из какого-либо одного материала. 
\end{enumerate}


\influence. Разработанные аналитические и численные методы для решения
уравнений Максвелла в рамках теории Ми и реализующий их
программный комплекс с использованием стахостической оптимизации
методом дифференциальной эволюции могут быть использованы при
проектировании, оптимизации и анализе (включая анализ предельно
достижимых рабочих характеристик) широкого спектра устройств,
функционирующих как в оптическом, так и микроволновом диапазоне.
Результаты полученные при изучении поглощени света наночастицами могут
быть использованы при разработке инновационных устройств
наноплазмоники, фотоактивных катализаторов, красителей, поглощающих
эмульсий и аэрозолей.

Результаты диссертационной работы использовались при выполнении
грантов Министерства образования и науки РФ
(проект 11.G34.31.0020, гос. задание 2014/190, задание 3.561.2014/K),
Правительства РФ (грант 074-U01), РФФИ (грант 15-57-45141 ИНД\verb+_+а).


\reliability\ полученных результатов обеспечивается методическим
подходом на каждом этапе работы. Работа оптимизатора была проверена на
наборе стандартных тестовых функций. Компьютерная реализация решения
задачи Ми была проверена на наборе тестовых задач, полученые
результаты были сопоставлены с результатами других авторов по теории
Ми для случаев однородной сферы и сферы с одним слоем
покрытия~\cite{Suzuki-2013, Bashevoy-2005}.  Случаи большего числа
слоёв в покрытии сравнивался с коммерческими пакетами моделирования,
использующих численные методы конечных разностей во временной области
(Lumerical FDTD), метод конечных элементов (Comsol) и метод конечных
интегралов (CST MWS). Результаты по исследованию маскирующих покрытий,
поглощению и рассеянию света наночастицами находятся в соответствии с
результатами, полученными другими авторами для похожих
систем~\cite{Semouchkina-2013, Alu-2014, Fan-2011}.

\probation\
Основные результаты работы докладывались~на:
перечисление основных конференций, симпозиумов и~т.\:п.

\contribution\ Автор принимал активное участие \ldots

%\publications\ Основные результаты по теме диссертации изложены в ХХ печатных изданиях~\cite{Sokolov,Gaidaenko,Lermontov,Management},
%Х из которых изданы в журналах, рекомендованных ВАК~\cite{Sokolov,Gaidaenko}, 
%ХХ --- в тезисах докладов~\cite{Lermontov,Management}.
 
\ifthenelse{\equal{\thebibliosel}{0}}{% Встроенная реализация с загрузкой файла через движок bibtex8
    \publications\ Основные результаты по теме диссертации изложены в XX печатных изданиях, 
    X из которых изданы в журналах, рекомендованных ВАК, 
    X "--- в тезисах докладов.%
}{% Реализация пакетом biblatex через движок biber
%Сделана отдельная секция, чтобы не отображались в списке цитированных материалов
    \begin{refsection}%
        \printbibliography[heading=countauthornotvak, env=countauthornotvak, keyword=biblioauthornotvak, section=1]%
        \printbibliography[heading=countauthorvak, env=countauthorvak, keyword=biblioauthorvak, section=1]%
        \printbibliography[heading=countauthorconf, env=countauthorconf, keyword=biblioauthorconf, section=1]%
        \printbibliography[heading=countauthor, env=countauthor, keyword=biblioauthor, section=1]%
        \publications\ Основные результаты по теме диссертации изложены в \arabic{citeauthor} печатных изданиях\nocite{bib1,bib2}, 
        \arabic{citeauthorvak} из которых изданы в журналах, рекомендованных ВАК\nocite{Ladutenko-cloak-2014,Ladutenko-Qabs-2015}, 
        \arabic{citeauthorconf} "--- в тезисах докладов\nocite{DD-14, MW-14}.%
    \end{refsection}
}
% При использовании пакета \verb!biblatex! для автоматического подсчёта
% количества публикаций автора по теме диссертации, необходимо
% их здесь перечислить с использованием команды \verb!\nocite!.
    


%%% Local Variables:
%%% mode: latex
%%% TeX-master: "../synopsis"
%%% End:
