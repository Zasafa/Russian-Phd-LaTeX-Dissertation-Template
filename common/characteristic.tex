{\actuality} Существование фундаментальных ограничений на скорость
обработки информации в традиционной электронике привело к бурному
развитию нанофотоники. В последние годы появилось большое количество
работ~\cite{Tame-quantum-plasmonics-2013,
  Javier-graphene-plasmonics-2014, Khurgin-loss-plasmonics-2015,
  He-tunable-terahertz-graphene-metamaterials-2015,
  Segal-meta-nonlinar-PhC-2015,
  Poddubny-hyperbolic-metamaterials-2013, Kildishev-metasurface-2013}
в этой области.  Высокая востребованность полученных результатов
связана с перспективами их практического применения и обусловлена
стремительным развитием нанотехнологий, что даёт возможность
экспериментальной проверки предлагаемых идей и подходов. Одной из
важных компонент нанофотоники являются наночастицы.  Кроме
нанофотоники наночастицы применяются для лечения
рака~\cite{Zhang-2010, Hirsch-2003}, различных методов диагностики в
медицине~\cite{Allain-2002}, разработки маскирующих суб-волновых
покрытий для видимого и микроволнового диапазонов~\cite{Qui-2009,
  Semouchkina-2013}, устройств плазмоники~\cite{Martin-2013,
  Alu-2005}, изучения тепловых свойств изоляторов~\cite{Xie-2013},
повышения эффективности солнечных элементов~\cite{Kameya-2011,
  Mann-2011}.  Всё это определяет актуальность данной работы.

% \underline{\textbf{Степень разработанности:}} TODO хороших инструметнов для
% изучения многослоек - мало, способов получать годные новые дизайны -
% того меньше.
  
{\aim} работы является % разработка общего подхода к оптимизации
% дизайнов многослойных сфер в рамках теории Ми, его последующая
% реализация в комплексе компьютерных программ, выявление
% закономерностей между дизайном многослойной сферы и её
исследование рассеяния и поглощения электромагнитных волн многослойными
сферическими наночастицами.

Для~достижения поставленной цели необходимо было решить следующие {\tasks}:
\begin{enumerate}
  \item Разработать алгоритм для вычисления рассеяния и поглощения в
    многослойных сферических объектах и реализовать его в комплексе программ.
  \item Выбрать и реализовать алгоритм оптимизации, подходящий для
    работы с произвольными параметрами модели.% , описываемой обобщённой
    % теорией Ми    
  \item Выявить основные закономерности взаимодействия с
    электромагнитной волной сферических маскирующих покрытий на
    основе диэлектриков.
  \item Исследовать эффект суперпоглощения света в многослойных
    сферических наночастицах.
\end{enumerate}

{\methods} В число основных инструментов, применяемых при анализе задач
рассеяния и поглощения плоской электромагнитной волны объектами со
сферической симметрией, входит теория Ми~\cite{Mie-1908}, суть которой
сводится к разложению полей в ряд по сферическим векторным
гармоникам. Эта теория была обобщена на случай многослойной сферы с
произвольным числом слоёв~\cite{Yang-2003, Pena-scattnlay-2009} и
модифицирована в настоящей работе. % , что позволило реализовать её в
% виде комплекса программ для проведения компьютерного моделирования

Для изучения оптических свойств многослойной наночастицы требуется
решать обратную задачу, т.е. определять дизайн многослойной сферы
(радиусы и комплексные показатели преломления составных слоёв) для
заданных характеристик рассеяния или поглощения. Для случая однородной
сферы подобная задача была частична решена, например, в дипольном
приближении были получены выражения для показателя преломления
обеспечивающего максимально достижимое значения поглощения для сферы
заданного размера~\cite{Grigoriev-2015}. Однако, для
общего случая многослойной сферы учёт вклада мультиполей высших
порядков и большое число параметров модели делает явное решение
труднореализуемым. Вместо этого в настоящей работе используется
метод стохастической оптимизации, а именно метод адаптивной
дифференциальной эволюции. Он позволяет численным образом проводить
оптимизацию дизайна многослойной сферы, обеспечивающего наилучшие
рабочие характеристики для каждого конкретного случая с учётом
фактических ограничений в выбранной предметной области.

%\vspace{5.5em}
{\novelty} В работе предложен и реализован новый подход к изучению
оптических свойств многослойных сферических наночастиц: совместное
использование метода адаптивной дифференциальной эволюции и теории Ми.
Высокая вычислительная производительность этого подхода позволила
выявить несколько новых физических эффектов связанных с рассеянием и
поглощением электромагнитной волны на многослойных сферических
наночастицах.

{\defpositions}
%Защищать цифры (уменьшили в два раза и т.д.).
\begin{enumerate}
  \item Предложен метод изучения экстремальных оптических свойств
    многослойных сферических наночастиц с помощью теории Ми и
    стохастической оптимизации.
  \item В задаче рассеяния плоской волны на многослойной сфере получены
    явные рекуррентные соотношения для коэффициентов Ми, выраженные
    через логарифмические производные функций Риккати-Бесселя для
    расчёта локальных полей.%  (TODO для
    % чего? для вычислени полей... + расширить (см. Балошин). + чем
    % отличается от просветляющих покрытий. + добавить инструмент
    % (программу) + семинар у Фёдорова (TODO показать ему готовый автореферат).)
  \item Рассеяние от объекта из идеального проводника может быть
    существенно уменьшено с помощью многослойного покрытия толщиной
    $0.15\lambda$ используя только изотропные диэлектрические
    материалы: в 2 и в 6 раз для объектов диаметром $1.5\lambda$ и
    $\lambda/1.5$ соответственно.
  \item % (TODO берем балошени без слов min-max-min)
    Среди множества оптимизированных дизайнов выявлена закономерность,
    позволяющая разрабатывать многослойные маскирующие сферические
    покрытия из изотропных материалов с $\varepsilon$ меньше единицы.

    % Использование изотропных материалов с $\varepsilon$ меньше единицы
    % позволяет создавать нерезонансные маскирующие покрытия, которые
    % уменьшают полное сечение рассеяния в 2 и более раза для объекта
    % из идеального проводника диаметром $1.5\lambda$ и содержат всего 3
    % слоя.

    %%%%%% Спектр- засада
  \item В трёхслойных частицах $Si/Ag/Si$ возможно вырождение
    мультипольных резонансов, приводящее к эффекту суперпоглощения,
    когда сечение поглощения оказывается больше, чем у однородной
    частицы того же размера из произвольного изотропного материала.
\end{enumerate}


{\influence}. Разработанные аналитические и численные методы для решения
уравнений Максвелла в рамках теории Ми и реализующий их
программный комплекс с использованием стохастической оптимизации
методом дифференциальной эволюции могут быть использованы при
проектировании, оптимизации и анализе (включая анализ предельно
достижимых рабочих характеристик) широкого спектра устройств,
функционирующих как в оптическом, так и микроволновом диапазоне.
Результаты полученные при изучении поглощения света наночастицами могут
быть использованы при разработке инновационных устройств
наноплазмоники, фотоактивных катализаторов, красителей, поглощающих
эмульсий и аэрозолей.

Результаты диссертационной работы использовались при выполнении
грантов Министерства образования и науки РФ
(проект 11.G34.31.0020, гос. задание 2014/190, задание 3.561.2014/K),
Правительства РФ (грант 074-U01), РФФИ (грант 15-57-45141 ИНД\verb+_+а).


{\reliability} полученных результатов обеспечивается методическим
подходом на каждом этапе работы. Работа оптимизатора была проверена на
наборе стандартных тестовых функций. Компьютерная реализация решения
задачи Ми была проверена на наборе тестовых задач, полученные
результаты были сопоставлены с результатами других авторов по теории
Ми для случаев однородной сферы и сферы с одним слоем
покрытия~\cite{Suzuki-2013, Bashevoy-2005}.  Случаи большего числа
слоёв в покрытии сравнивался с коммерческими пакетами моделирования,
использующих численные методы конечных разностей во временной области
(Lumerical FDTD), метод конечных элементов (Comsol) и метод конечных
интегралов (CST MWS). Результаты по исследованию маскирующих покрытий,
поглощению и рассеянию света наночастицами находятся в соответствии с
результатами, полученными другими авторами для похожих
систем~\cite{Semouchkina-2013, Alu-2014, Fan-2011}.

{\probation}
Основные результаты работы докладывались~на:
перечисление основных конференций, симпозиумов и~т.\:п.

{\contribution} Автор принимал активное участие \ldots

%\publications\ Основные результаты по теме диссертации изложены в ХХ печатных изданиях~\cite{Sokolov,Gaidaenko,Lermontov,Management},
%Х из которых изданы в журналах, рекомендованных ВАК~\cite{Sokolov,Gaidaenko}, 
%ХХ --- в тезисах докладов~\cite{Lermontov,Management}.
 
\ifthenelse{\equal{\thebibliosel}{0}}{% Встроенная реализация с загрузкой файла через движок bibtex8
    \publications\ Основные результаты по теме диссертации изложены в XX печатных изданиях, 
    X из которых изданы в журналах, рекомендованных ВАК, 
    X "--- в тезисах докладов.%
}{% Реализация пакетом biblatex через движок biber
%Сделана отдельная секция, чтобы не отображались в списке цитированных материалов
    \begin{refsection}%
        \printbibliography[heading=countauthornotvak, env=countauthornotvak, keyword=biblioauthornotvak, section=1]%
        \printbibliography[heading=countauthorvak, env=countauthorvak, keyword=biblioauthorvak, section=1]%
        \printbibliography[heading=countauthorconf, env=countauthorconf, keyword=biblioauthorconf, section=1]%
        \printbibliography[heading=countauthor, env=countauthor, keyword=biblioauthor, section=1]%
        \publications\ Основные результаты по теме диссертации изложены в \arabic{citeauthor} печатных изданиях\nocite{bib1,bib2}, 
        \arabic{citeauthorvak} из которых изданы в журналах, рекомендованных ВАК\nocite{Ladutenko-cloak-2014,Ladutenko-Qabs-2015}, 
        \arabic{citeauthorconf} "--- в тезисах докладов\nocite{DD-14, MW-14}.%
    \end{refsection}
}
% При использовании пакета \verb!biblatex! для автоматического подсчёта
% количества публикаций автора по теме диссертации, необходимо
% их здесь перечислить с использованием команды \verb!\nocite!.
    


%%% Local Variables:
%%% mode: latex
%%% TeX-master: "../synopsis"
%%% End:
