{\actuality} Теория Ми входит в число основных инструментов
применяемых при анализе задач рассеяния и поглощения плоской
электромагнитной волны сферическими объектами.  В настоящее время
теория Ми была обобщена на случай многослойных сфер, что позволяет
использовать её в целом ряде прикладных задач, таких как лечение рака,
различные методы диагностики в медицине, разработка маскирующих
суб-волновые покрытий для видимого и микроволнового диапазонов,
устройств плазмоники, изучение тепловых свойств изоляторов, для
повышения эффективности солнечных элементов и так далее.  Как правило,
при решении подобных прикладных задач возникает необходимость
оптимизации дизайна многослойной сферы (радиусов и материальных
параметров составных слоёв), обеспечивающего наилучшие рабочие
характеристики для каждого конкретного случая с учётом фактических
ограничений в предметной области.
% Обзор, введение в тему, обозначение места данной работы в мировых исследованиях и~т.\:п.

\aim\ данной работы является создание общего подхода к проектированию
оптимизированных дизайнов многослойных сфер в рамках теории Ми и
реализация этого подхода в комплексе компьютерных программ.

Для~достижения поставленной цели необходимо было решить следующие {\tasks}:
\begin{enumerate}
  \item Выбрать и реализовать универсальный алгоритм оптимизации.
  \item Доработать численный алгоритм теории Ми и реализовать эти
    доработоки в комплексе программ.
  \item Выявить основные закономерности взаимодействия с
    электромагнитной волной маскирующих покрытий как на основе только
    диэлектриков, так и при использовании метаматериалов.
  \item Исследовать поглощение света в многослойных сферических наночастицах.
\end{enumerate}

\defpositions
\begin{enumerate}
\item Программный комплекс, реализующий параллельный алгоритм
  стохастической оптимизации методом адаптивной дифференциальной
  эволюции.
\item Математическая модель и её программная реализация для расчёта
  полей в рамках теории Ми для многослойных сфер.
\item Маскирующие сферические покрытия на основе диэлектриков и метаматериалов.
\item Эффект суперпоглощения в сферических наночастицах.
\end{enumerate}
Положения соответствуют пунктам 2,3,4,5,8 паспорта специальности
05.13.18 -- <<Математическое моделирование, численные методы и
комплексы программ>> (%
%% 1. Разработка новых математических методов моделирования объектов и
%% явлений.
%2.
Развитие качественных и приближенных аналитических методов
исследования математических моделей.
%%3. 
Разработка, обоснование и тестирование эффективных вычислительных
методов с применением современных компьютерных технологий.
%4. 
Реализация эффективных численных методов и алгоритмов в виде
комплексов проблемно-ориентированных программ для проведения
вычислительного эксперимента.
%%5. 
Комплексные исследования научных и технических проблем с
применением современной технологии математического моделирования и
вычислительного эксперимента.
%6. Разработка новых математических методов и алгоритмов проверки
% адекватности математических моделей объектов на основе данных
% натурного эксперимента.
%%7. Разработка новых математических методов и алгоритмов
%% интерпретации натурного эксперимента на основе его математической
%% модели.
%8. 
Разработка систем компьютерного и имитационного моделирования%
) по физико-математическим наукам (преобладают математические методы в
качестве аппарата исследований; получены результаты в виде новых
математических методов, вычислительных алгоритмов и новых
закономерностей, характеризующих изучаемые объекты).  \novelty
\begin{enumerate}
  \item Впервые были получены явные реккурентные соотношения для
    коэффициентов Ми в многослойной сфере, выраженные через
    логарифмические производные функций Риккати-Бесселя. 
  \item Впервые метод дифференциальной эволюции был применён
    для изучения маскирующих сферических покрытий, показана высокая
    производительность метода.
  \item Было выполнено оригинальное исследование поглощения света
    наночастицами в режиме вырождения резонансых откликов мультиполей.
\end{enumerate}

\influence. Разработанные аналитические и численные методы для решения
уравнений Максвелла в рамках теории Ми, а так же реализующий их
программный комплекс с использованием стахостической оптимизации
методом дифференциальной эволюции могут быть использованы при
проектировании, оптимизации и анализе (включая анализ предельно
достижимых рабочих характеристик) широкого спектра устройств,
работающих как в оптическом, так и микроволновом диапазоне. Результаты
полученные при изучении поглощени света наночастицами могут быть
использованы при разработке инновационных фотоактивных катализаторов,
красителей, поглощающих эмульсий и аэрозолей.

Результаты диссертационной работы использовались при выполнении
грантов Министерства образования и науки РФ
(проект 11.G34.31.0020, гос. задание 2014/190, задание 3.561.2014/K),
Правительства РФ (грант 074-U01), РФФИ (грант 15-57-45141 ИНД\verb+_+а).


\reliability\ полученных результатов обеспечивается методическим
подходом на каждом этапе работы. Работа оптимизатора была проверена на
наборе стандартных тестовых функций. Аналитические результаты работы
были проверены в системе компьютерной алгебры (IPython). Компьютерная
реализация решения была проверена на наборе тестовых задач. Результаты
находятся в соответствии с результатами, полученными другими авторами
по теории Ми для случаев однородной сферы и сферы с одним слоем
покрытия.  Случаи большего числа слоёв в покрытии сравнивался с
коммерческими пакетами моделирования, использующих численные методы
конечных разностей во временной области (Lumerical FDTD), метод
конечных элементов (Comsol) и метод конечных интегралов (CST
MWS). Результаты по исследованию маскирующих покрытий и поглощения
света наночастицами находятся в соответствии с результатами,
полученными другими авторами для похожих систем. 

\probation\
Основные результаты работы докладывались~на:
перечисление основных конференций, симпозиумов и~т.\:п.

\contribution\ Автор принимал активное участие \ldots

%\publications\ Основные результаты по теме диссертации изложены в ХХ печатных изданиях~\cite{Sokolov,Gaidaenko,Lermontov,Management},
%Х из которых изданы в журналах, рекомендованных ВАК~\cite{Sokolov,Gaidaenko}, 
%ХХ --- в тезисах докладов~\cite{Lermontov,Management}.

\ifthenelse{\equal{\thebibliosel}{0}}{% Встроенная реализация с загрузкой файла через движок bibtex8
    \publications\ Основные результаты по теме диссертации изложены в XX печатных изданиях, 
    X из которых изданы в журналах, рекомендованных ВАК, 
    X "--- в тезисах докладов.%
}{% Реализация пакетом biblatex через движок biber
%Сделана отдельная секция, чтобы не отображались в списке цитированных материалов
    \begin{refsection}%
        \printbibliography[heading=countauthornotvak, env=countauthornotvak, keyword=biblioauthornotvak, section=1]%
        \printbibliography[heading=countauthorvak, env=countauthorvak, keyword=biblioauthorvak, section=1]%
        \printbibliography[heading=countauthorconf, env=countauthorconf, keyword=biblioauthorconf, section=1]%
        \printbibliography[heading=countauthor, env=countauthor, keyword=biblioauthor, section=1]%
        \publications\ Основные результаты по теме диссертации изложены в \arabic{citeauthor} печатных изданиях\nocite{bib1,bib2}, 
        \arabic{citeauthorvak} из которых изданы в журналах, рекомендованных ВАК\nocite{vakbib1,vakbib2}, 
        \arabic{citeauthorconf} "--- в тезисах докладов\nocite{confbib1,confbib2}.%
    \end{refsection}
}
    

