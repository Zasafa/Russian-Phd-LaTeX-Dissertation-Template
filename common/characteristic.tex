{\actuality} Существование фундаментальных ограничений на скорость
обработки информации в традиционной электронике привело к бурному
развитию нанофотоники. В последние годы появилось большое количество
работ~\cite{Tame-quantum-plasmonics-2013,
  Javier-graphene-plasmonics-2014, Khurgin-loss-plasmonics-2015,
  He-tunable-terahertz-graphene-metamaterials-2015,
  Segal-meta-nonlinar-PhC-2015,
  Poddubny-hyperbolic-metamaterials-2013, Kildishev-metasurface-2013}
в этой области.  Высокая востребованность результатов, полученных в
этих и многих других работах, связана с перспективами их практического
применения; стремительное развитие нанотехнологий даёт возможность
экспериментальной проверки предлагаемых идей и подходов. Часто в
устройствах нанофотоники основная функциональность реализуется с
помощью наночастиц. Кроме нанофотоники, многослойные наночастицы применяются
для лечения рака~\cite{Zhang-2010, Hirsch-2003}, различных методов
диагностики в медицине~\cite{Allain-2002}, повышения эффективности
солнечных элементов~\cite{Kameya-2011, Mann-2011}, разработки
маскирующих субволновых покрытий видимого и микроволнового
диапазонов~\cite{Qui-2009, Semouchkina-2013}, устройств
плазмоники~\cite{Martin-2013, Alu-2005}.  Всё это определяет
актуальность и цель данной работы.

% \underline{\textbf{Степень разработанности:}} хороших инструметнов для
% изучения многослоек --- мало, способов получать годные новые дизайны ---
% того меньше.
{\aim} работы является % разработка общего подхода к оптимизации
% дизайнов многослойных сфер в рамках теории Ми, его последующая
% реализация в комплексе компьютерных программ, выявление
% закономерностей между дизайном многослойной сферы и её
исследование рассеяния и поглощения электромагнитных волн многослойными
сферическими наночастицами.
Для~достижения поставленной цели необходимо решить следующие {\tasks}:
\begin{enumerate}
  \item Разработать алгоритм вычисления рассеяния и поглощения
    электромагнитных волн
    многослойными сферическими объектами и реализовать его в комплексе программ.
  \item Выбрать и реализовать алгоритм оптимизации, подходящий для
    работы с произвольными параметрами модели.% , описываемой обобщённой
    % теорией Ми    
  \item Выявить основные закономерности взаимодействия с
    электромагнитной волной сферических маскирующих покрытий на
    основе диэлектриков.
  \item Исследовать эффект суперпоглощения света в многослойных
    сферических наночастицах, когда из-за вырождения мультипольных
    резонансов сечение поглощения оказывается больше, чем у однородной
    частицы из произвольного изотропного материала того же общего
    размера.
\end{enumerate}

{\methods} В число основных инструментов, применяемых при анализе задач
рассеяния и поглощения плоской электромагнитной волны наночастицами,
входит теория Ми~\cite{Mie-1908}, суть которой
сводится к разложению полей в ряд по сферическим векторным
гармоникам. Эта теория была обобщена на случай многослойной сферы с
произвольным числом слоёв~\cite{Yang-2003, Pena-scattnlay-2009} и
модифицирована в настоящей работе. % , что позволило реализовать её в
% виде комплекса программ для проведения компьютерного моделирования

Для изучения оптических свойств многослойной наночастицы требуется
решать обратную задачу, т.е. определять дизайн такой частицы
(радиусы и комплексные показатели преломления составных слоёв) при
заданных характеристиках рассеяния или поглощения. В случае однородной
наночастицы подобная задача была частично решена: например, в дипольном
приближении было получено выражение для эффективного значения
диэлектрической проницаемости, обеспечивающего максимально достижимое
поглощение сферой заданного
размера~\cite{Grigoriev-2015}. Однако для общего случая многослойной
сферы учёт вклада мультиполей высших порядков и большое число
параметров модели делают явное решение труднореализуемым. Вместо этого
в настоящей работе используется стохастическая оптимизация, а
именно метод адаптивной дифференциальной эволюции. Он позволяет
численным образом проводить оптимизацию дизайна многослойной сферы,
обеспечивающего наилучшие рабочие характеристики для каждого
конкретного случая с учётом фактических ограничений в выбранной
предметной области.

{\novelty} В задаче рассеяния плоской волны на многослойной сфере
впервые получены явные соотношения для коэффициентов Ми внутри сферы,
выраженные через логарифмические производные функций Риккати-Бесселя в
виде обратной рекуррентной последовательности. В работе предложен и
реализован новый подход к изучению оптических свойств многослойных
сферических наночастиц: совместное использование метода адаптивной
дифференциальной эволюции и теории Ми.  Высокая вычислительная
производительность этого подхода позволила выявить несколько новых
физических эффектов, связанных с рассеянием и поглощением
электромагнитной волны на многослойных сферических наночастицах. Были
получены дизайны тонких (по сравнению с длиной волны) маскирующих
покрытий из изотропных диэлектриков, изучены их свойства, рассмотрены
причины, приводящие к возникновению маскирующего эффекта. Для случая
размещения наночастицы в диэлектрической среде выявлена
закономерность, позволяющая разрабатывать эффективные маскирующие
трёхслойные сферические покрытия.  Универсальность предлагаемого
подхода позволяет изучать не только рассеяние, но и поглощение
наночастицами. Аналитическими методами был впервые продемонстрирован
эффект суперпоглощения в сферических наночастицах специального
дизайна. Важно, что это удалось показать для случая модели дисперсии изотропных
материалов, соответствующей экспериментальным
данным.

% \vspace{5.5em}
{\defpositions}  %Защищать цифры (уменьшили в два раза и т.д.).
\mynobreakpar\begin{enumerate}
  \item Совместное применение теории Ми и стохастической оптимизации
    позволяет изучать экстремальные оптические свойства многослойных
    сферических наночастиц.
  \item В задаче рассеяния плоской волны на многослойной сфере
    коэффициенты Ми внутри сферы могут быть явно выражены в виде
    обратной рекуррентной последовательности через логарифмические
    производные функций Риккати-Бесселя, что обеспечивает улучшенную
    численную устойчивость расчёта. %  + чем
    % отличается от просветляющих покрытий. + добавить инструмент
    % (программу) + семинар у Фёдорова (TODO показать ему готовый
    % автореферат).)    
  \item Рассеяние на объекте из идеального проводника можно
    существенно уменьшить с помощью тонкого многослойного покрытия,
    используя только изотропные диэлектрические материалы: для
    объектов диаметром $1.5\lambda$ в два раза, для диаметра
    $\lambda/1.5$ - в 6 раз. Оптимальная толщина покрытия определяется
    доступным диапазоном используемых материальных параметров.
  \item 
    При анализе покрытий из множества слоёв равной толщины
    была выявлена закономерность, позволяющей разрабатывать
    сферические покрытия из трёх слоёв различной толщины, а именно,
    для достижения аналогичной эффективности маскировки достаточно
    чередования двух материалов, один из которых использует
    $\varepsilon<1$. 
    % Использование изотропных материалов с $\varepsilon$ меньше единицы
    % позволяет создавать нерезонансные маскирующие покрытия, которые
    % уменьшают полное сечение рассеяния в 2 и более раза для объекта
    % из идеального проводника диаметром $1.5\lambda$ и содержат всего 3
    % слоя.

    %%%%%% Спектр --- засада
  \item На примере структуры $Si/Ag/Si$ показана возможность
    вырождения мультипольных резонансов в сферических трёхслойных
    наночастицах, приводящее к эффекту суперпоглощения, когда сечение
    поглощения оказывается больше, чем у однородной частицы того же
    размера из произвольного изотропного материала.
\end{enumerate}
% В папке Documents можно ознакомиться в решением совета из Томского ГУ
% в файле \verb+Def_positions.pdf+, где обоснованно даются рекомендации
% по формулировкам защищаемых положений. 


{\influence}. Разработанные аналитические и численные методы для решения
уравнений Максвелла в рамках теории Ми и реализующий их
программный комплекс с использованием стохастической оптимизации
методом дифференциальной эволюции могут быть использованы при
проектировании, оптимизации и анализе (включая анализ предельно
достижимых рабочих характеристик) широкого спектра устройств,
функционирующих как в оптическом, так и микроволновом диапазоне.
Результаты, полученные при изучении поглощения света наночастицами, могут
быть применены при разработке инновационных устройств
наноплазмоники, фотоактивных катализаторов, красителей, поглощающих
эмульсий и аэрозолей.

Результаты диссертационной работы использовались при выполнении
грантов Министерства образования и науки РФ
(проект 11.G34.31.0020, гос. задание 2014/190, задание 3.561.2014/K),
Правительства РФ (грант 074-U01), РФФИ (грант 15-57-45141 ИНД\verb+_+а).


{\reliability} полученных результатов обеспечивается методическим
подходом на каждом этапе работы, сравнением с данными других авторами
и сопоставлением с результатами моделирования в других программах.
Работа оптимизатора была проверена на наборе стандартных тестовых
функций. Компьютерная реализация решения задачи Ми была проверена на
наборе тестовых примеров, полученные результаты были сопоставлены с
данными других авторов для случаев однородной сферы и сферы с одним
слоем покрытия~\cite{Suzuki-2013, Bashevoy-2005}.  Случаи большего
числа слоёв в покрытии сравнивались с коммерческими пакетами
моделирования, использующими численные методы конечных разностей во
временной области (Lumerical FDTD), метод конечных элементов (Comsol)
и метод конечных интегралов (CST MWS). Результаты по исследованию
маскирующих покрытий, поглощению и рассеянию света наночастицами
находятся в соответствии с данными, полученными другими авторами для
похожих систем~\cite{Semouchkina-2013, Alu-2014, Fan-2011}.

{\probation} Основные результаты работы докладывались~на конференциях
<<Дни дифракции>> (СПб, 2014 и 2016), <<Электроника и микроэлектроника
СВЧ>> (СПб, 2014), Metanano (Анапа, 2016) и семинарах в университете
ИТМО, ФТИ им.~А.Ф.~Иоффе, ФГУП <<Крыловский государственный научный
центр>> и в университете им. Жана Монне (Сент-Этьен, Франция).

{\contribution} Все результаты данной диссертационной работы получены
автором лично, их анализ проводился при его непосредственном участии.
Кроме того, автор самостоятельно провёл все работы, связанные с
программированием алгоритма стохастической оптимизации.  Программа
Scattnlay, используемая для расчётов по теории Ми, была полностью
переработана автором диссертации совместно с автором оригинальной
программы Ovidio Pe\~{n}a-Rodr\'{i}guez (Политехнический университет
Мадрида, Испания).

%\publications\ Основные результаты по теме диссертации изложены в ХХ печатных изданиях~\cite{Sokolov,Gaidaenko,Lermontov,Management},
%Х из которых изданы в журналах, рекомендованных ВАК~\cite{Sokolov,Gaidaenko}, 
%ХХ --- в тезисах докладов~\cite{Lermontov,Management}.

\ifnumequal{\value{bibliosel}}{0}{% Встроенная реализация с загрузкой файла через движок bibtex8
    \publications\ Основные результаты по теме диссертации изложены в XX печатных изданиях, 
    X из которых изданы в журналах, рекомендованных ВАК, 
    X "--- в тезисах докладов.%
}{% Реализация пакетом biblatex через движок biber
%Сделана отдельная секция, чтобы не отображались в списке цитированных материалов
    \begin{refsection}%
      \printbibliography[heading=countauthorvak, env=countauthorvak,
      keyword=biblioauthorvak, section=1]%
      \printbibliography[heading=countauthorconf, env=countauthorconf,
      keyword=biblioauthorconf, section=1]%
      \printbibliography[heading=countauthornotvak,
      env=countauthornotvak, keyword=biblioauthornotvak, section=1]%
      \printbibliography[heading=countauthor, env=countauthor,
      keyword=biblioauthor, section=1]%
      \publications\ Основные результаты по теме диссертации изложены
      в \arabic{citeauthor} печатных изданиях,
      \arabic{citeauthorvak}~из которых изданы в журналах,
      рекомендованных ВАК, \arabic{citeauthorconf}~в тезисах
      конференций, получено одно государственное свидетельство о
      регистрации программы для ЭВМ~№2014611568 от 5~февраля 2014~г. в
      Федеральной службе по интеллектуальной
      собственности.\nocite{Ladutenko-Qabs-2015,Ladutenko-cloak-2014,Markovich-FDTD-2013,Ladutenko-FDTD-2012,Metanano-16,DD-16,DD-14,MW-14,Ladutenko-Svidetelstvo-2014}%
    \end{refsection}
}
% При использовании пакета \verb!biblatex! для автоматического подсчёта
% количества публикаций автора по теме диссертации, необходимо
% их здесь перечислить с использованием команды \verb!\nocite!.

 


%%% Local Variables:
%%% mode: latex
%%% TeX-master: "../synopsis"
%%% End:
