{\actuality} Теория Ми описывает взаимодействие плоской
электромагнитной волны со сферической частицей.  Тем не менее, эта
теория представляет значительный интерес и в наши дни, так как она
входит в число основных инструментов применяемых при анализе задач
рассеяния и поглощения сферическими объектами.  В настоящее время
теория Ми была обобщена на случай многослойных сфер, что позволяет
использовать её в целом ряде прикладных задач, таких как лечение рака,
различные методы диагностики в медицине, разработка маскирующих
суб-волновые покрытий для видимого и микроволнового диапазонов,
устройств плазмоники, изучение тепловых свойств изоляторов, для
повышения эффективности солнечных элементов и многих других.
% Обзор, введение в тему, обозначение места данной работы в мировых исследованиях и~т.\:п.


 \aim\ данной работы является \ldots

Для~достижения поставленной цели необходимо было решить следующие {\tasks}:
\begin{enumerate}
  \item Исследовать, разработать, вычислить и~т.\:д. и~т.\:п.
  \item Исследовать, разработать, вычислить и~т.\:д. и~т.\:п.
  \item Исследовать, разработать, вычислить и~т.\:д. и~т.\:п.
  \item Исследовать, разработать, вычислить и~т.\:д. и~т.\:п.
\end{enumerate}

\defpositions
\begin{enumerate}
  \item Программный комплекс, реализующий алгоритм стохастической
    оптимизации методом адаптивной дифференциальной эволюции.
  \item Математическая модель и её программная реализация для расчёта
    полей в рамках теории Ми для многослойных сфер.
  \item Маскирующие сферические покрытия на основе диэлектриков и метаматериалов.
  \item Эффект суперпоглощения в сферических наночастицах.
\end{enumerate}
Положения соответствуют пунктам 2,3,4,5,8 паспорта специальности
05.13.18 -- <<Математическое моделирование, численные методы и
комплексы программ>> по физико-математическим наукам (преобладают
математические методы в качестве аппарата исследований; получены
результаты в виде новых математических методов, вычислительных
алгоритмов и новых закономерностей, характеризующих изучаемые
объекты).
%% 1. Разработка новых математических методов моделирования объектов и
%% явлений.
%2. Развитие качественных и приближенных аналитических методов
% исследования математических моделей.
%%3. Разработка, обоснование и тестирование эффективных вычислительных
%% методов с применением современных компьютерных технологий.
%4. Реализация эффективных численных методов и алгоритмов в виде
% комплексов проблемно-ориентированных программ для проведения
% вычислительного эксперимента.
%%5. Комплексные исследования научных и технических проблем с
%% применением современной технологии математического моделирования и
%% вычислительного эксперимента.
%6. Разработка новых математических методов и алгоритмов проверки
% адекватности математических моделей объектов на основе данных
% натурного эксперимента.
%%7. Разработка новых математических методов и алгоритмов
%% интерпретации натурного эксперимента на основе его математической
%% модели.
%8. Разработка систем компьютерного и имитационного моделирования.

\novelty
\begin{enumerate}
  \item Впервые \ldots
  \item Впервые \ldots
  \item Было выполнено оригинальное исследование \ldots
\end{enumerate}

\influence\ \ldots

\reliability\ полученных результатов обеспечивается \ldots \ Результаты находятся в соответствии с результатами, полученными другими авторами.

\probation\
Основные результаты работы докладывались~на:
перечисление основных конференций, симпозиумов и~т.\:п.

\contribution\ Автор принимал активное участие \ldots

%\publications\ Основные результаты по теме диссертации изложены в ХХ печатных изданиях~\cite{Sokolov,Gaidaenko,Lermontov,Management},
%Х из которых изданы в журналах, рекомендованных ВАК~\cite{Sokolov,Gaidaenko}, 
%ХХ --- в тезисах докладов~\cite{Lermontov,Management}.

\ifthenelse{\equal{\thebibliosel}{0}}{% Встроенная реализация с загрузкой файла через движок bibtex8
    \publications\ Основные результаты по теме диссертации изложены в XX печатных изданиях, 
    X из которых изданы в журналах, рекомендованных ВАК, 
    X "--- в тезисах докладов.%
}{% Реализация пакетом biblatex через движок biber
%Сделана отдельная секция, чтобы не отображались в списке цитированных материалов
    \begin{refsection}%
        \printbibliography[heading=countauthornotvak, env=countauthornotvak, keyword=biblioauthornotvak, section=1]%
        \printbibliography[heading=countauthorvak, env=countauthorvak, keyword=biblioauthorvak, section=1]%
        \printbibliography[heading=countauthorconf, env=countauthorconf, keyword=biblioauthorconf, section=1]%
        \printbibliography[heading=countauthor, env=countauthor, keyword=biblioauthor, section=1]%
        \publications\ Основные результаты по теме диссертации изложены в \arabic{citeauthor} печатных изданиях\nocite{bib1,bib2}, 
        \arabic{citeauthorvak} из которых изданы в журналах, рекомендованных ВАК\nocite{vakbib1,vakbib2}, 
        \arabic{citeauthorconf} "--- в тезисах докладов\nocite{confbib1,confbib2}.%
    \end{refsection}
}
    

