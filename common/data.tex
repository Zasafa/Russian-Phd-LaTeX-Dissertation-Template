%%% Основные сведения %%%
\newcommand{\thesisAuthor}             % Диссертация, ФИО автора
{%
    \texorpdfstring{% \texorpdfstring takes two arguments and uses the first for (La)TeX and the second for pdf
        Ладутенко Константин Сергеевич % так будет отображаться на титульном листе или в тексте, где будет использоваться переменная
    }{%
        Ладутенко, Константин Сергеевич% эта запись для свойств pdf-файла. В таком виде, если pdf будет обработан программами для сбора библиографических сведений, будет правильно представлена фамилия.
    }%
}
\newcommand{\thesisAuthorShort}             % Диссертация, ФИО автора инициалами
{К.С.~Ладутенко}

\newcommand{\thesisUdk}                % Диссертация, УДК
{535[341+36+434]}
% 535.341   5.3.5.3.4.1   Коэффициент поглощения. Коэффициент экстинкции и аналогичные характеристики поглощения
% 535.36   5.3.5.3.6   Рассеяние света. Диффузия
% 535.434   5.3.5.4.3.4   Дифракция в коллоидных растворах. Дифракция, вызываемая малыми частицами. Мутность

%УДК 535 Оптика
%УДК 535.3   Распространение световых лучей. Излучение. Отражение. Преломление. Поглощение
%УДК 535.34  Поглощение. Спектры поглощения
%    535.341 Коэффициент поглощения. Коэффициент экстинкции и аналогичные характеристики поглощения
%    535.36  Рассеяние света. Диффузия
%УДК 535.4   Интерференция. Дифракция. Дифракционное рассеяние
%УДК 535.43  Рассеяние при дифракции. Свечение Тиндаля
%    535.434 Дифракция в коллоидных растворах. Дифракция, вызываемая малыми частицами. Мутность

\newcommand{\thesisTitleBoth}          % Диссертация, название
% {Моделирование взаимодействия оптимизированной многослойной сферы с
% плоской электромагнитной волной}
% {Рассеяние и поглощение электромагнитных волн многослойными
% сферическими порытиями}
{Рассеяние и поглощение электромагнитных волн многослойными
сферическими наночастицами}

\newcommand{\thesisTitle}              % Диссертация, название
{\texorpdfstring{\MakeUppercase{\thesisTitleBoth}}{\thesisTitleBoth}}

\newcommand{\thesisSpecialtyNumberBoth}    % Диссертация, специальность, номер
{01.04.05}
\newcommand{\thesisSpecialtyNumber}    % Диссертация, специальность, номер
{\texorpdfstring{\thesisSpecialtyNumberBoth}{\thesisSpecialtyNumberBoth}}

\newcommand{\thesisSpecialtyTitleBoth}     % Диссертация, специальность, название
{Оптика}
\newcommand{\thesisSpecialtyTitle}     % Диссертация, специальность, название
{\texorpdfstring{\thesisSpecialtyTitleBoth}{\thesisSpecialtyTitleBoth}}


% \newcommand{\thesisSpecialtyNumberBothSecond}    % Диссертация, специальность, номер
% {05.13.18}
% \newcommand{\thesisSpecialtyNumberSecond}    % Диссертация, специальность, номер
% {\texorpdfstring{\todo{\thesisSpecialtyNumberBothSecond}}{\thesisSpecialtyNumberBothSecond}}

% \newcommand{\thesisSpecialtyTitleBothSecond}     % Диссертация, специальность, название
% { Математическое моделирование, численные методы и комплексы программ}
% \newcommand{\thesisSpecialtyTitleSecond}     % Диссертация, специальность, название
% {\texorpdfstring{\todo{\thesisSpecialtyTitleBothSecond}}{\thesisSpecialtyTitleBothSecond}}



\newcommand{\thesisDegree}             % Диссертация, научная степень
{кандидата физико-математических наук}
\newcommand{\thesisDegreeShort}        % Диссертация, ученая степень, краткая запись
{канд. физ.-мат. наук}
\newcommand{\thesisCity}               % Диссертация, город защиты
{Санкт-Петербург}
\newcommand{\thesisYear}               % Диссертация, год защиты
{2017}
\newcommand{\thesisOrganization}       % Диссертация, организация
{Федеральное государственное автономное образовательное учреждение высшего образования <<Санкт-Петербургский национальный исследовательский университет информационных технологий, механики и оптики>>}
\newcommand{\thesisOrganizationShort}  % Диссертация, краткое название организации для доклада
{Университет ИТМО}

\newcommand{\thesisInOrganization}       % Диссертация, организация в предложном падеже: Работа выполнена в ...
{федеральном государственном автономном образовательном учреждении высшего образования <<Санкт-Петербургский национальный исследовательский университет информационных технологий, механики и оптики>>}

\newcommand{\supervisorFio}            % Научный руководитель, ФИО
{Белов Павел Александрович}
\newcommand{\supervisorRegalia}        % Научный руководитель, регалии
{доктор физико-математических наук}
\newcommand{\supervisorFioShort}            % Научный руководитель, ФИО
{П.А.~Белов}
\newcommand{\supervisorRegaliaShort}        % Научный руководитель, регалии
{д-р~физ.-мат.~наук}

% Жаров А.А. Нижний новгород опонент 1 докторо
% Федянин (один из двух, лучше второй)
% Залипаев (из Крылова)?

% У андрея богданова спросить про ведующую организацию в Политехе.



\newcommand{\opponentOneFio}           % Оппонент 1, ФИО
{Климов Василий Васильевич}

\newcommand{\opponentOneRegalia}       % Оппонент 1, регалии
{доктор физико-математических наук}
\newcommand{\opponentOneJobPlace}      % Оппонент 1, место работы
{Федеральное государственное унитарное предприятие <<Всероссийский
научно-исследовательский институт автоматики им. Н.Л. Духова>> (ФГУП
<<ВНИИА>>) }
\newcommand{\opponentOneJobPost}       % Оппонент 1, должность
{главный научный сотрудник}

\newcommand{\opponentTwoFio}           % Оппонент 2, ФИО
{Бельтюков Ярослав Михайлович}
%{\todo{ Залипаев Виктор Васильевич}} % Федянин?
\newcommand{\opponentTwoRegalia}       % Оппонент 2, регалии
{кандидат физико-математических наук}
\newcommand{\opponentTwoJobPlace}      % Оппонент 2, место работы
{Федеральное государственное бюджетное учреждение науки
Физико-технический институт им.~А.Ф.~Иоффе Российской академии наук}

%{\todo{ФГУП <<Крыловский научный центр>>}}
\newcommand{\opponentTwoJobPost}       % Оппонент 2, должность
{исполняющий обязанности научного сотрудника}
%{\todo{должность}}
%{\todo{старший научный сотрудник}}

\newcommand{\leadingOrganizationTitle} % Ведущая организация, дополнительные строки
{Автономная некоммерческая образовательная организация высшего
профессионального образования <<Сколковский институт науки и
технологий>>}% Гиппиус 
%Юрий Капитонов спбгу

\newcommand{\defenseDate}              % Защита, дата
{<<16>> ноября
2017~г. в~17~часов 10 минут}
\newcommand{\defenseCouncilNumber}     % Защита, номер диссертационного совета
{Д\,212.227.02}
\newcommand{\defenseCouncilTitle}      % Защита, учреждение диссертационного совета
{\thesisInOrganization}
%{\todo{Название учреждения}}
\newcommand{\defenseCouncilAddress}    % Защита, адрес учреждение диссертационного совета
{197101, Санкт-Петербург, Кронверкский пр., д.~49, ауд.~331}

\newcommand{\defenseCouncilPhone}      % Телефон для справок
{+7~(812)~232-4284}
%{\todo{+7~(0000)~00-00-00}}

\newcommand{\defenseSecretaryFio}      % Секретарь диссертационного совета, ФИО
{Фёдоров\par
Анатолий\par
Валентинович}
%{\todo{Фамилия Имя Отчество}}
\newcommand{\defenseSecretaryRegalia}  % Секретарь диссертационного совета, регалии
{доктор физико-математических наук}
%{\todo{д-р~физ.-мат. наук}}            % Для сокращений есть ГОСТы, например: ГОСТ Р 7.0.12-2011 + http://base.garant.ru/179724/#block_30000

\newcommand{\synopsisLibrary} % Автореферат, название библиотеки
{университета ИТМО и на сайте http://fppo.ifmo.ru/?page1=16\&page2=52\&page\_d=1\&page\_d2=159525 }
%{университета ИТМО и на сайте https://physics.ifmo.ru/phd/2017/ladutenko}
%{\todo{Название библиотеки}}
\newcommand{\synopsisDate}             % Автореферат, дата рассылки
%{<<$\,$\rule[-2pt]{0.7cm}{0.5pt}$\,$>> \rule[-2pt]{3cm}{0.5pt}~2017 года}
{<<$\,$28$\,$>> сентября~2017 года}

% To avoid conflict with beamer class use \providecommand
\providecommand{\keywords}%            % Ключевые слова для метаданных PDF диссертации и автореферата
{теория Ми, суперпоглощение, стохастическая оптимизация, маскирующие покрытия}
