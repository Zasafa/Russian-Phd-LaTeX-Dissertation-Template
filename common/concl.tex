%% Согласно ГОСТ Р 7.0.11-2011:
%% 5.3.3 В заключении диссертации излагают итоги выполненного исследования, рекомендации, перспективы дальнейшей разработки темы.
%% 9.2.3 В заключении автореферата диссертации излагают итоги данного исследования, рекомендации и перспективы дальнейшей разработки темы.
\begin{enumerate}
  \item В задаче рассеяния плоской волны на многослойной сфере
    получены явные рекуррентные соотношения для коэффициентов Ми
    внутри сферы, выраженные через логарифмические
    производные функций Риккати-Бесселя.  Эти соотношения были
    добавлены в компьютерную программу, выполняющую вычисления в
    рамках задачи Ми.
  \item Предложен метод изучения экстремальных оптических свойств
    многослойных сферических наночастиц с помощью теории Ми и
    стохастической оптимизации. Высокая вычислительная
    производительность этого подхода позволила выявить несколько новых
    физических эффектов, связанных с рассеянием и поглощением
    электромагнитной волны на многослойных сферических наночастицах.
  \item Были получены дизайны тонких (по сравнению с длиной волны)
    маскирующих покрытий из изотропных диэлектриков, изучены их
    свойства, рассмотрены причины, приводящие к возникновению
    маскирующего эффекта.  Обнаружен пороговый характер уменьшения
    рассеяния в зависимости от толщины покрытия.
  \item % (TODO берем балошени без слов min-max-min)
    Среди разнообразных оптимизированных дизайнов маскирующих покрытий из
    изотропных материалов с $\varepsilon$ меньше единицы, состоящих из
    множества слоёв равной толщины, выявлена закономерность,
    позволяющая разрабатывать эффективные трёхслойные сферические
    покрытия с разными толщинами слоёв. Дополнительной особенностью
    таких покрытий является значительное увеличение области спектра, в
    которой наблюдается эффект маскировки, при сравнении с покрытиями
    из диэлектриков. 
    %%%%%% Спектр --- засада
  \item В трёхслойных частицах $Si/Ag/Si$ возможно вырождение
    мультипольных резонансов, приводящее к эффекту суперпоглощения,
    когда сечение поглощения оказывается больше, чем у однородной
    частицы того же размера из произвольного изотропного
    материала. Максимальная эффективность поглощения в
    рассматриваемой системе была получена для небольших двухслойных
    частиц с преобладающей ролью электрического дипольного резонанса.
\end{enumerate}
