\subsection*{Общая характеристика работы}

\newcommand{\actuality}{\underline{\textbf{Актуальность темы.}}}
\newcommand{\aim}{\underline{\textbf{Целью}}}
\newcommand{\tasks}{\underline{\textbf{задачи}}}
\newcommand{\defpositions}{\underline{\textbf{Основные положения, выносимые на~защиту:}}}
\newcommand{\novelty}{\underline{\textbf{Научная новизна:}}}
\newcommand{\influence}{\underline{\textbf{Практическая значимость}}}
\newcommand{\reliability}{\underline{\textbf{Достоверность}}}
\newcommand{\probation}{\underline{\textbf{Апробация работы.}}}
\newcommand{\contribution}{\underline{\textbf{Личный вклад.}}}
\newcommand{\publications}{\underline{\textbf{Публикации.}}}

{\actuality} 
% Актуальность - необходимо уметь контролировать рассеяние и поглощение,
% есть невидимость. Добавить 5 ссылок. Актуально сделать маскирующие
% покрытие на основе диэлектриков. 

В последние годы появилось большое количество работ по
нанофотонике~\cite{Tame-quantum-plasmonics-2013,
  Javier-graphene-plasmonics-2014, Khurgin-loss-plasmonics-2015,
  He-tunable-terahertz-graphene-metamaterials-2015,
  Segal-meta-nonlinar-PhC-2015,
  Poddubny-hyperbolic-metamaterials-2013, Kildishev-metasurface-2013}.
Высокая актуальность полученных результатов связана с перспективами их
практического применения и обусловлена стремительным развитием
нанотехнологий, что даёт возможность экспериментальной проверки
предлагаемых идей и подходов. Среди прочих, стоит отметить вопрос о
взаимодействии света с многослойной сферической наночастицей. Он
рассматривается в ряде прикладных задач, таких как: лечение
рака~\cite{Zhang-2010, Hirsch-2003}, различные методы диагностики в
медицине~\cite{Allain-2002}, разработка маскирующих суб-волновых
покрытий для видимого и микроволнового диапазонов~\cite{Qui-2009,
  Semouchkina-2013}, устройства плазмоники~\cite{Martin-2013,
  Alu-2005}, изучение тепловых свойств изоляторов~\cite{Xie-2013},
повышение эффективности солнечных элементов~\cite{Kameya-2011,
  Mann-2011} и так далее. Всё вместе это обуславливает актуальность
настоящей работы, в которой сперва излагается общий принцип,
позволяющий управлять рассеянием и поглощением электромагнитных волн
многослойными сферическими наночастицами, а потом идёт апробация на
частных примерах: минимизация рассеяния от идеально проводящей сферы
(частичная невидимость) и управление поглощением плазмонной частицы
$Si/Ag/Si$.

\underline{\textbf{Основные методы исследования.}}
Теория Ми~\cite{Mie-1908} входит в число основных инструментов применяемых
при анализе задач рассеяния и поглощения плоской электромагнитной
волны сферическими объектами. Эта теория была обобщённа на случай
многослойной сферы с произвольным числом слоёв~\cite{Yang-2003,
  Pena-scattnlay-2009} и доработана в настоящей работе, что позволило
реализовать её в виде комплекса программ для проведения компьютерного
моделирования. Достоинством теории является используемое ей разложение
поля по сферическим векторным гармоникам, что позволяет разделить
вклад в общее поле от электрического и магнитного дипольного
резонанса, а так же вклад резонансов квадруполя
и мультиполей более высокого порядка. Таким образом, становится
возможен анализ спектрального отклика многослойной сферы в зависимости
от её параметров (размеров и показателей преломления слоёв). Например,
в ряде случаев удаётся совместить в спектре рассеяния положение
нескольких резонансов (например, электрических дипольного и
квадрупольного), что создаёт эффект
суперрассеяния~\cite{Fan-2010,Fan-2011}. Аналогичный эффект
суперпоглощения подробно рассмотрен в настоящей работе.


 Как правило, при их решении возникает
необходимость оптимизации дизайна многослойной сферы (радиусов и
материальных параметров составных слоёв), обеспечивающего наилучшие
рабочие характеристики для каждого конкретного случая с учётом
фактических ограничений в предметной области.


\aim\ данной работы является разработка общего подхода к оптимизации
дизайнов многослойных сфер в рамках теории Ми, его последующая
реализация в комплексе компьютерных программ, выявление
закономерностей между дизайном многослойной сферы и её оптическими
свойствами.

Для~достижения поставленной цели необходимо было решить следующие {\tasks}:
\begin{enumerate}
  \item Разработать алгоритм для вычисления рассеяния и поглощения в
    многослойных сферических объектах и реализовать его в комплексе программ.
  \item Выбрать и реализовать алгоритм оптимизации, подходящий для
    работы с произвольными параметрами модели, описываемой обобщённой
    теорией Ми.
  \item Выявить основные закономерности взаимодействия с
    электромагнитной волной сферических маскирующих покрытий на
    основе диэлектриков.
  \item Исследовать эффект суперпоглощения света в многослойных
    сферических наночастицах.
\end{enumerate}



\defpositions
\begin{enumerate}
  \item Получены и реализованны в комплексе программ явные
    реккурентные соотношения для коэффициентов Ми в объёме
    многослойной сферы, выраженные через логарифмические производные
    функций Риккати-Бесселя увеличивающие численную стабильность.  
  
  \item Использование тонкого (размер мишени к размеру покрытия) диэлектриких многослойных покрытий позволяет
    уменьшить рассеяние от идеальное мишени в два раза.
  \item Использовать диэлектрикого порытия для небольшого объекта
    позволяет уменьшить рассеяние в 6 раз.

  \item TODO Использование алгоритма стохастической оптимизации методом
    адаптивной дифференциальной эволюции для решения задачи Ми
    позволяет выявлять семейства дизайнов с заранее заданными
    электромагнитными свойствами.

  \item Защищать цифры (уменьшили в два раза и т.д.). Показано, что
    маскирующие сферические покрытия из диэлектриков могут быть
    сконструированы, используя волноводоподобный эффект.  В этом
    случае при распространении внутри покрытия поле отстает по фазе от
    невозмущённой падающей волны на величину, кратную $2\pi$.
  \item Обнаружено семейство маскирующих сферических порытий из
    диэлектрических изотропных метаматериалов, реализующих эффект
    волнового обтекания.  Для получения заметного эффекта достаточно
    трёх слоёв в покрытии.
  \item В трёхслойных частицах $Si/Ag/Si$ возможно вырождение
    резонансных мультипольных откликов, приводящее к эффекту
    суперпоглощения, когда сечение поглощение оказывается больше, чем
    у bulk частицы. 
  \end{enumerate}

Положения соответствуют пункту 1 паспорта специальности 01.04.05 --
<<Оптика>> (Волновая (физическая) оптика. Интерференция, дифракция,
поляризация, когерентность света) по физико-математическим
наукам (представлены результаты фундаментальных исследований).

%\vspace{5.5em}
\novelty Используем диэлектрики для маскировки, использовать
оптимизация. Есть суперпоглощения.
\begin{enumerate}
  \item Впервые были получены явные реккурентные соотношения для
    коэффициентов Ми в многослойной сфере, выраженные через
    логарифмические производные функций Риккати-Бесселя. 
  \item Впервые метод дифференциальной эволюции был применён
    для изучения маскирующих сферических покрытий, показана высокая
    производительность метода.
  \item Было выполнено оригинальное исследование поглощения света
    наночастицами в режиме вырождения резонансых мультипольных откликов.
\end{enumerate}

\influence. Разработанные аналитические и численные методы для решения
уравнений Максвелла в рамках теории Ми, а так же реализующий их
программный комплекс с использованием стахостической оптимизации
методом дифференциальной эволюции могут быть использованы при
проектировании, оптимизации и анализе (включая анализ предельно
достижимых рабочих характеристик) широкого спектра устройств,
работающих как в оптическом, так и микроволновом диапазоне. Результаты
полученные при изучении поглощени света наночастицами могут быть
использованы при разработке инновационных устройств наноплазмоники,
фотоактивных катализаторов, красителей, поглощающих эмульсий и
аэрозолей.

Результаты диссертационной работы использовались при выполнении
грантов Министерства образования и науки РФ
(проект 11.G34.31.0020, гос. задание 2014/190, задание 3.561.2014/K),
Правительства РФ (грант 074-U01), РФФИ (грант 15-57-45141 ИНД\verb+_+а).


\reliability\ полученных результатов обеспечивается методическим
подходом на каждом этапе работы. Работа оптимизатора была проверена на
наборе стандартных тестовых функций. Аналитические результаты работы
были проверены в системе компьютерной алгебры (IPython). Компьютерная
реализация решения была проверена на наборе тестовых
задач. Аналитические результаты находятся в соответствии с
результатами, полученными другими авторами по теории Ми для случаев
однородной сферы и сферы с одним слоем покрытия.  Случаи большего
числа слоёв в покрытии сравнивался с коммерческими пакетами
моделирования, использующих численные методы конечных разностей во
временной области (Lumerical FDTD), метод конечных элементов (Comsol)
и метод конечных интегралов (CST MWS). Результаты по исследованию
маскирующих покрытий и поглощения света наночастицами находятся в
соответствии с результатами, полученными другими авторами для похожих
систем.

\probation\
Основные результаты работы докладывались~на:
перечисление основных конференций, симпозиумов и~т.\:п.

\contribution\ Автор принимал активное участие \ldots

%\publications\ Основные результаты по теме диссертации изложены в ХХ печатных изданиях~\cite{Sokolov,Gaidaenko,Lermontov,Management},
%Х из которых изданы в журналах, рекомендованных ВАК~\cite{Sokolov,Gaidaenko}, 
%ХХ --- в тезисах докладов~\cite{Lermontov,Management}.
 
\ifthenelse{\equal{\thebibliosel}{0}}{% Встроенная реализация с загрузкой файла через движок bibtex8
    \publications\ Основные результаты по теме диссертации изложены в XX печатных изданиях, 
    X из которых изданы в журналах, рекомендованных ВАК, 
    X "--- в тезисах докладов.%
}{% Реализация пакетом biblatex через движок biber
%Сделана отдельная секция, чтобы не отображались в списке цитированных материалов
    \begin{refsection}%
        \printbibliography[heading=countauthornotvak, env=countauthornotvak, keyword=biblioauthornotvak, section=1]%
        \printbibliography[heading=countauthorvak, env=countauthorvak, keyword=biblioauthorvak, section=1]%
        \printbibliography[heading=countauthorconf, env=countauthorconf, keyword=biblioauthorconf, section=1]%
        \printbibliography[heading=countauthor, env=countauthor, keyword=biblioauthor, section=1]%
        \publications\ Основные результаты по теме диссертации изложены в \arabic{citeauthor} печатных изданиях\nocite{bib1,bib2}, 
        \arabic{citeauthorvak} из которых изданы в журналах, рекомендованных ВАК\nocite{Ladutenko-cloak-2014,Ladutenko-Qabs-2015}, 
        \arabic{citeauthorconf} "--- в тезисах докладов\nocite{DD-14, MW-14}.%
    \end{refsection}
}
% При использовании пакета \verb!biblatex! для автоматического подсчёта
% количества публикаций автора по теме диссертации, необходимо
% их здесь перечислить с использованием команды \verb!\nocite!.
    

 % Характеристика работы по структуре во введении и в автореферате не отличается (ГОСТ Р 7.0.11, пункты 5.3.1 и 9.2.1), потому её загружаем из одного и того же внешнего файла, предварительно задав форму выделения некоторым параметрам

%Диссертационная работа была выполнена при поддержке грантов ...

%\underline{\textbf{Объем и структура работы.}} Диссертация состоит из~введения, четырех глав, заключения и~приложения. Полный объем диссертации \textbf{ХХХ}~страниц текста с~\textbf{ХХ}~рисунками и~5~таблицами. Список литературы содержит \textbf{ХХX}~наименование.

%\newpage
\subsection*{Содержание работы}
Во \underline{\textbf{введении}} обосновывается актуальность
исследований, проводимых в рамках данной диссертационной работы,
приводится обзор научной литературы по изучаемой проблеме,
формулируется цель, ставятся задачи работы, сформулированы научная
новизна и практическая значимость представляемой работы.

\underline{\textbf{Первая глава}} посвящена выбору универсального
алгоритма оптимизации и вопросам его практической
реализации. Сложность выбора обусловлена огромным количеством методов
оптимизации, а так же большим числом разновидностей каждого
метода. При выборе метода применительно к задаче Ми были использованны
следующие предпосылки:
\begin{itemize}
\item Несмотря на то, что решение Ми является аналитическим и
  выражается в виде разложения в ряд по сферическим векторным
  гармоникам, одновременное нахождение производных для зависимости от
  радиуса и материального параметра оказывается громоздким даже в
  случае однородной сферы, что тем более верно для случая
  произвольного числа сферических слоёв. В связи с чем метод
  оптимизации не должен требовать для своей работы нахождения
  производных оптимизируемой функции. Это особено актуально, в случае,
  когда одновременно оптимизируются и толщина, и показатель
  преломления каждого слоя, или, например, оптимизируемая величина
  берётся в нескольких точках спектра одновременно..
\item Решение образовано быстро-осциллирующими функциями и, как
  следствие, будет содержать большое количество локальных
  экстремумов. Таким образом, алгоритмы оптимизации, требующие особого
  отношения к подобным случаям, оказываются заведомо менее
  производительными.
\item Параметры оптимизации (например, толщина и показатель
  преломления каждого слоя), а так же оптимизируемая величина являются
  вещественными числами.
\end{itemize}

Всё вместе это позволяет ограничить выбор стахостическими методами,
среди которых наиболее распространёнными являются генетические
алгоритмы, методы роя части и методы дифференциальной эволюции.  Эти
алгоритмы используют метод <<проб и ошибок>>. Несколько пробных
решений (или индивидов) генерируются случайным образом и многократно
улучшаются с надеждой найти некое удовлетворительное решение. Качество
решения оценивается целевой функцией, возникшей из задачи, которую
предстоит оптимизировать. Полная группа индивидов называется
популяцией. Состояние популяции на конкретном шаге итерации называется
поколением. Переход между поколениями осуществляется в соответствии с
рядом относительно простых правил, которые составляют сущность
определённого алгоритма.

Генетические алгоритмы обычно рассматривают вещественные числа в виде
набора битов. В отличие от них, методы роя частиц и методы
дифференциальной эволюции могут работать в непрерывном пространстве
вещественных входных параметров естественным образом (используя
возможность сложения и вычитания векторов пробных решений), что делает
их гораздо более удобными для решения физических
задач. Производительность этих алгоритмов зависит от правильного
выбора значений некоторых внутренних параметров
алгоритма. Использование адаптивных версий алгоритмов упрощает задачу
оптимизации: значения внутренних параметров настраиваются
автоматически при переходе между поколениями. Как правило, адаптивным
алгоритмам нужно гораздо меньше (более чем на порядок) итераций, чем
неадаптивным, чтобы добиться того же результата оптимизации.

В настоящей работе был использован алгоритм JADE+ с улучшенной
скоростью скрещивания (по алгоритму PMCRADE), который является
адаптивным вариантом алгоритма дифференциальной эволюции. Он имеет
явное преимущество перед адаптивной оптимизацией методом роя частиц в
ряде стандартных тестов.

 картинку можно добавить так:
\begin{figure}[ht] 
  \center
  \includegraphics [scale=0.27] {latex}
  \caption{Подпись к картинке.} 
  \label{img:latex}
\end{figure}

Формулы в строку без номера добавляются так:
\[ 
  \lambda_{T_s} = K_x\frac{d{x}}{d{T_s}}, \qquad
  \lambda_{q_s} = K_x\frac{d{x}}{d{q_s}},
\]

\underline{\textbf{Вторая глава}} посвящена исследованию 

\underline{\textbf{Третья глава}} посвящена исследованию 

В \underline{\textbf{четвертой главе}} приведено описание 

В \underline{\textbf{заключении}} приведены основные результаты работы, которые заключаются в следующем:
%% Согласно ГОСТ Р 7.0.11-2011:
%% 5.3.3 В заключении диссертации излагают итоги выполненного исследования, рекомендации, перспективы дальнейшей разработки темы.
%% 9.2.3 В заключении автореферата диссертации излагают итоги данного исследования, рекомендации и перспективы дальнейшей разработки темы.
\begin{enumerate}
  \item Предложен метод изучения экстремальных оптических свойств
    многослойных сферических наночастиц с помощью теории Ми и
    стохастической оптимизации. Высокая вычислительная
    производительность этого подхода позволила выявить несколько новых
    физических эффектов, связанных с рассеянием и поглощением
    электромагнитной волны на многослойных сферических наночастицах.
  \item В задаче рассеяния плоской волны на многослойной сфере
    получены явные рекуррентные соотношения для коэффициентов Ми в
    расчёте локальных полей, выраженные через логарифмические
    производные функций Риккати-Бесселя.  Эти соотношения были
    добавлены в компьютерную программу, выполняющую вычисления в
    рамках задачи Ми.
  \item Рассеяние от объекта из идеального проводника можно
    существенно уменьшить с помощью многослойного покрытия толщиной
    $0.15\lambda$, используя только изотропные диэлектрические
    материалы: в 2 и в 6 раз для объектов диаметром $1.5\lambda$ и
    $\lambda/1.5$ соответственно. Обнаружен пороговый характер
    уменьшения рассеяния в зависимости от толщины покрытия.
  \item % (TODO берем балошени без слов min-max-min)
    Среди разнообразных оптимизированных дизайнов маскирующих покрытий из
    изотропных материалов с $\varepsilon$ меньше единицы, состоящих из
    множества слоёв равной толщины, выявлена закономерность,
    позволяющая разрабатывать эффективные трёхслойные сферические
    покрытия с разными толщинами слоёв. Дополнительной особенностью
    таких покрытий является значительное увеличение области спектра, в
    которой наблюдается эффект маскировки, при сравнении с покрытиями
    из диэлектриков. 
    %%%%%% Спектр --- засада
  \item В трёхслойных частицах $Si/Ag/Si$ возможно вырождение
    мультипольных резонансов, приводящее к эффекту суперпоглощения,
    когда сечение поглощения оказывается больше, чем у однородной
    частицы того же размера из произвольного изотропного
    материала. Максимальная эффективность поглощения в
    рассматриваемой системе была получена для небольших двухслойных
    частиц с преобладающей ролью электрического дипольного резонанса.
\end{enumerate}



%\newpage
\renewcommand{\refname}{\large Публикации автора по теме диссертации}

\ifthenelse{\equal{\thebibliosel}{0}}{% Встроенная реализация с загрузкой файла через движок bibtex8
    \nocite{*}
}{% Реализация пакетом biblatex через движок biber
}

\insertbiblioauthor                          % Подключаем Bib-базы
%\insertbibliofull
