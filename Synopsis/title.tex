\thispagestyle{empty}
\doublehyphendemerits=500000
\vspace{0pt plus1fill} %число перед fill = кратность относительно некоторого расстояния fill, кусками которого заполнены пустые места
\begin{center}
  \includegraphics[height=1.5cm]{itmo-logo}
  \hfill
  \begin{minipage}{.34\linewidth}
    \begin{center}
      \large{На правах рукописи}    \\
      \vspace{1em}
    \includegraphics[height=1.3cm]{personal-signature}      
    \end{center}
  \end{minipage}
\end{center}

\vspace{0pt plus3fill} %число перед fill = кратность относительно некоторого расстояния fill, кусками которого заполнены пустые места
\begin{center}
\textbf {\large \thesisAuthor}
\end{center}

\vspace{0pt plus3fill} %число перед fill = кратность относительно некоторого расстояния fill, кусками которого заполнены пустые места
\begin{center}
\textbf {\Large \thesisTitle}

\vspace{0pt plus3fill} %число перед fill = кратность относительно некоторого расстояния fill, кусками которого заполнены пустые места
{\large Специальность \thesisSpecialtyNumber\ "---\par <<\thesisSpecialtyTitle>>}\par
% {\large Специальность \thesisSpecialtyNumberSecond\ "---\par <<\thesisSpecialtyTitleSecond>>}\par

\vspace{0pt plus1.5fill} %число перед fill = кратность относительно некоторого расстояния fill, кусками которого заполнены пустые места
\Large{Автореферат}\par
\large{диссертации на соискание учёной степени\par \thesisDegree}
\end{center}

\vspace{0pt plus4fill} %число перед fill = кратность относительно некоторого расстояния fill, кусками которого заполнены пустые места
\begin{center}
{\large{\thesisCity\ "--- \thesisYear}}
\end{center}

\newpage
% оборотная сторона обложки
\thispagestyle{empty}
{\doublehyphendemerits=1000000000
\noindent Работа выполнена в \thesisInOrganization

\par\bigskip
\noindent%
\begin{tabularx}{\textwidth}{@{}lX@{}}
    Научный руководитель:   & \supervisorRegalia\par
                              \textbf{\supervisorFio}
    \vspace{0.013\paperheight}\\
    Официальные оппоненты:  &
        \textbf{\opponentOneFio,}\par
        \opponentOneRegalia,\par
        \opponentOneJobPlace,\par
        \opponentOneJobPost\par
            \vspace{0.01\paperheight}
        \textbf{\opponentTwoFio,}\par
        \opponentTwoRegalia,\par
        \opponentTwoJobPlace,\par
        \opponentTwoJobPost
    \vspace{0.013\paperheight} \\
    Ведущая организация:    & \leadingOrganizationTitle
\end{tabularx}  
\par\bigskip

\noindent Защита состоится \defenseDate~на~заседании диссертационного совета \defenseCouncilNumber~при \defenseCouncilTitle~по адресу: \defenseCouncilAddress.

\vspace{0.015\paperheight}
\noindent С диссертацией можно ознакомиться в библиотеке
\synopsisLibrary. % TODO вставить ссылку на дисер на сайте ИТМО
}

% \vspace{0.017\paperheight}
% \noindent Отзывы на автореферат в двух экземплярах, заверенные печатью учреждения, просьба направлять по адресу: \defenseCouncilAddress, ученому секретарю диссертационного совета~\defenseCouncilNumber.

\vspace{0.015\paperheight}
\noindent{Автореферат разослан \synopsisDate.}

%\noindent Телефон для справок: \defenseCouncilPhone.

\vspace{0.015\paperheight}
\par\bigskip
\noindent%
\begin{tabularx}{\textwidth}{@{}%
>{\raggedright\arraybackslash}b{18em}
>{\centering\arraybackslash}X
b{6em}
@{}}
    Ученый секретарь\par
    диссертационного совета\par
    \defenseCouncilNumber,\par
    \defenseSecretaryRegalia
    &
    \includegraphics[width=2.3cm]{secretary-signature}
    &
    \defenseSecretaryFio
\end{tabularx} 
